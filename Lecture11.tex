\documentclass[mathserif,aspectratio=1610]{beamer}
%\documentclass[mathserif,handout,aspectratio=1610]{beamer}
%\usepackage{beamerthemeshadow}

\input{$HOME/tex/inputs/defsSkript}%$
\input{$HOME/tex/inputs/styleBeamerVkoekMa}%$
%\input{styleBeamerVkoekMa}%$

\usepackage{graphicx}

%\newcommand{\pathDiscrChoice}{$HOME/vorlesungen/Verkehrsoekonometrie_Ma/discrChoice_cc_Levmar} %$
\newcommand{\pathDiscrChoice}{figsDiscr} 

%##############################################################

\begin{document}
%###################################################
\frame{  %title layout
%###################################################
\makePale{1.00}{0.50}{0.70}{1.40}{1.40}

\placebox{0.50}{0.42}{
  \figSimple{1.60\textwidth}{figsIOM/AKWfig.png}}
%makePale{opacity}{centerXrel}{centerYrel}{wrel}\{hrel} 
\makePale{0.80}{0.50}{0.70}{1.40}{1.40}

\placebox{0.55}{0.89}{\parbox{0.45\textwidth}{\myheading{
11 \ Input-Output Models\\ \hspace{1em} and Life-Cycle Assessment}}}

\placebox{0.43}{0.14}{\parbox{0.8\textwidth}{
{\large
\bi
\item 11.1. Input-Output Model (IOM) of Leontief
\item 11.2. Life-Cycle Assessment (LCA)
\item 11.3. Combination: Econometric Input-Output LCA (EIO-LCA)

\ei
}
}}

}




\subsection{11.1. The IOM of Leontief}

%##############################################################
\frame{\frametitle{11.1. Input-Output Model (IOM) of Leontief: Motivation}
%##############################################################

\vspace{-1em}
\fig{1.0\textwidth}{figsIOM/AKWfig.png}
\vspace{-0.5em}

\bi
\item Atomic power plants do not have any direct \COii\ emissions
\pause \item However, what are the \emph{effective} emission considering all
  involved processes recursively?
\ei
}


%##############################################################
\frame{\frametitle{Input-output model: problem statement}
%##############################################################

\begin{center}
\parbox{0.8\textwidth}{
\bi
\item In a modern economy, nearly everything is connected to ``the
  rest'' of the economy.\\[1ex]
\pause \item \emph{Wanted:} a quantitative description of the flows of
material, 
  services, and information between the different parts of an economy.
\pause \item The \bfdef{input-output model (IOM) of Leontief} tackles
this problem by 
  making several assumptions:\\[1ex]
\bi
\item Every material or service is associated with a
  certain \bfdef{sector}\\[1ex]
\pause \item To make all flows (kg, \euro, bytes, ...) commensurable, the
  common unit is a \emph{monetary unit}, e.g., \euro\\[1ex]
\pause \item The whole system is \emph{linear} and \emph{deterministic}: douple input means double output.
  Particularly, there is no economy of scale\\[1ex]
\pause \item The whole system is in the \bfdef{steady state}, e.g., there are no
  temporal changes (constant supply and demand); storage (if
  applicable)  is
  neither built up nor depleted.
\ei
\ei
}
\end{center}

}

%##############################################################
\frame{\frametitle{Specification of the IOM of Leontief}
%##############################################################

Linear, deterministic coupling of $n$ \bfdef{sectors} and an
\bfdef{end consumer}  in
the steady state: 

\maindm{
x_i=y_i+\sum\limits_{j=1}^n x_{ij} =y_i+\sum\limits_{j=1}^n A_{ij} x_j.
}

\bi
\item $x_i$: Total output of sector $i$ in \euro{} or other monetary
  units per time unit\\[1ex]
\pause \item $y_i$: Flow of products/services of sector $i$ to the end
  consumers (and to sectors that are not explicitely considered)\\[1ex]
\pause \item $x_{ij}$: Flow from sector $i$ to $j$: Sector $j$ needs a supply
  $x_{ij}$ from sector $i$ to maintain the steady state and to ensure
  a constant 
  supply $y_j$ to the end consumer\\[1ex]
\pause \item $A_{ij}=x_{ij}/x_j$: \bfdef{IO coefficient}
  reflecting linearity: In order to
  produce one unit, sector $j$ needs $A_{ij}$ units from all the other
  sectors $i$, including the own.
\ei


}

%##############################################################
\frame{\frametitle{Visualisation of the flows 
generated by atomic power plants}
%##############################################################
\vspace{-0.5ex}
\fig{0.95\textwidth}{figsIOM/AKWflussdiag_eng.png}
}


%##############################################################
\frame{\frametitle{Total production for a given consumer's supply}
%##############################################################

IOM equation in vector-matrix notation:
\maindm{\vec{x}=\m{A}\cdot \vec{x}+\vec{y}}
\bi
\item $\vec{x}= (x_1,x_2, \cdots, x_n)\tr$ \bfdef{production vector}
\item $\vec{y}= (y_1,y_2, \cdots, y_n)\tr$ \bfdef{supply vector}
\item $\m{A}=(A_{ij}), i,j=1 \cdots n$ \bfdef{IOM coefficient matrix}
\ei
\vspace{1em}
Solving for $\vec{x}$ by writing $(\m{1}-\m{A})\vec{x}=\vec{y}$:
\maindm{\vec{x}=(\m{1}-\m{A})^{-1}\vec{y} \equiv \m{B}\vec{y} }
\bi
\item $\m{B}=(\m{1}-\m{A})^{-1}$ 
\bfdef{coefficient matrix of the final demand}
\ei
}


%##############################################################
\frame{\frametitle{Meaning of the matrix of the final
  demand \m{B}}
%##############################################################

\bi
\item $B_{ij}$ denotes the needed total production from sector $i$ in
  order to deliver one unit of $j$ to the end consumer (or the not
  considered sectors) in the steady state\\[1em]
\item $\m{B}$ includes all indirect effect \emph{in an infinite
  recursion} as can be seen from the Taylor expansion:
\ei
\vspace{1ex}

\bdm
\m{B}=\left(\m{1}-\m{A}\right)^{-1}
= \m{1}+\m{A}+\m{A}^2+\m{A}^3+\cdots
= \sum\limits_{j=0}^{\infty}\m{A}^j
\edm

}


%##############################################################
\frame{\frametitle{Example: 1=transportation sector, 2=vehicle
  construction } 
%##############################################################


{\small
\bdm
B_{11}= 1
\visible<2->{+A_{11}}
\visible<3->{+\sum\limits_{k=1}^2 A_{1k}A_{k1}}
\visible<4->{+\sum\limits_{k=1}^2\sum\limits_{l=1}^2A_{1k}A_{kl}A_{l1} + \ \ldots}
\edm

\bi
\item 1: Transportation of the passengers (``end consumers'')
\visible<2->{\item $A_{11}$: 
  The drivers, conductors, and the administrative staff of
the transportation companies need transportation themselves}
\visible<3->{\item $A_{11}^2$: 
  The transport of employees of the transportation companies induces
  additional traffic, hence the need for additional employees to scale
  up the supply accordingly}
\visible<3->{\item $A_{12}A_{21}$: 
To manage operations, the transport sector must offer aditional
transportation for the commutes of the workers/employees of the
vehicle making sector ($A_{12}$), so they can provide additional 
vehicles ($A_{21}$) needed by the transportation sector to maintain the steady
  state. }
\visible<4->{\item $A_{11}A_{12}A_{21}$: 
Since also the employees of
the transportation companies need transportation ($A_{11}$), even more
transportation supply ($A_{12}$) must be offered to the
employees of the vehicle making companies to get the additionally 
needed vehicles ($A_{21}$)
\item $\cdots$}
\ei
}


}

%##############################################################
\frame{\frametitle{Questions} 
%##############################################################

\bi
\itemAsk \colAsk{Argue that a national economy with sectors $i$ satisfying 
$\sum_jA_{ij}x_j>x_i$ would not be sustainable or needs external help
  (``GDR'').}

\pause\itemAnswer\colAnswer{In such an economy, sector $i$  
must deliver more units to operate itself ($A_{ii}x_i$) and the
 other sectors 
 ($A_{ij}x_j$) than this sector produces in total ($x_i$).}

\pause\itemAsk \colAsk{Give reasons why all $A_{ij}$ and $B_{ij}$ are $\ge 0$ and
  $B_{ii}\ge 1$.}
\pause\itemAnswer\colAnswer{Since sectors \emph{need} products and
services from other sectors. }

\pause\itemAsk \colAsk{Assume that the external demand $y_k$ for
  products/services of 
  sector $k$ suddenly increases by $r_k=\unit[1]{\%}$ (e.g., driven by
  politics). Give a 
  general expression for the
  percentaged increase of the GDP in
  order to re-attain the steady state.}

\pause\itemAnswer\colAnswer{The change of the demand vector is
    given by $\Delta \vec{y}=(0,..,r_ky_k,0,...)\tr$ and the change of
the production vector components by
$\Delta x_i=\sum_jB_{ij}y_j=r_kB_{ik}y_k$. Hence, the change of the 
total GDP is given by $\Delta x=\sum_i\Delta x_i=r_k\sum_i B_{ik}y_k$
and the old GDP itself by
$x=\sum_ix_i=\sum_i\sum_jB_{ij}y_j$. Finally, the
percentage increase of the total GDP  is given by $\Delta x/x$}
\ei


}




%##############################################################
\frame{\frametitle{Questions (ctnd.)} 
%##############################################################

\bi
\pause\itemAsk \colAsk{Give some additional elements and concepts
  needed to make the 
  IOM  dynamic}
%\ei}\end{document}

\pause\itemAnswer\colAnswer{
After a sudden change of the demand,
  the demand vector $\vec{y}$ is no longer balanced against the
  available production $(\m{1}-\m{A})\vec{x}$ and the excess demand or
  supply is balanced by emptying or filling the stores. If the economy is
  \bfdef{demand-driven (Keynes)}, this also induces ramping up/down
  the production. In the simplest case, the rate of change of the
  production is proportional to the excess demand,
\bdm
\abl{x_i}{t}
=\frac{1}{\tau_i}\bigg[y_i(t)-\big( (\m{1}-\m{A})\vec{x}\big)_i\bigg] 
\edm
where $\tau_i$ is the time the sector $i$ needs to adapt to changing
demands.
}
\\[1ex]

\pause\itemAsk \colAsk{Give some additional elements and concepts
  needed to introduce nonlinearity reflecting the economy of scale}

\pause\itemAnswer\colAnswer{
In an \bfdef{economy of scale}, the IO coefficients become
  smaller with the number of produced units of the target sector which
  may be modelled, 
  e.g., by
\bdm
A_{ij}(x_j)=\frac{A_{ij}(0)}{1+x_j/x_{j0}}
\edm
where $x_{j0}$ is the production quantity where significal scale
effects set in.
}
\ei

}

\subsection{11.2. Life-Cycle Assessment}

%##############################################################
\frame{\frametitle{11.2. Life-Cycle Assessment (LCA): Motivation} 
%##############################################################

\bi
\item The IOM reflects a \emph{snapshot} of \emph{all products} of a
  national economy in the \emph{steady  state}

\pause \item Sometimes, it is more instructive or relevant to consider
  \emph{the total 
  lifetime} of \emph{a single product} in a \emph{time dependent way}
  by assessing production, operation, and destruction/recycling of
  this product. 

\pause \item This is formalized by the methods of \bfdef{Life-Cycle
  Assessment (LCA)} (German: \bfdef{\"Okobilanz}).

\pause \item However, LCA only considers first-order indirect effects, e.g.,
  \COii\ emissions caused by electric vehicles through the
  \COii\ footprint of electricity production

\pause \item The class of \bfdef{Econometric Input-Output (EIO)
  LCA models} combines both approaches. 
\ei
}



\providecommand{\tily}{\tilde{y}}
\providecommand{\tilC}{\tilde{C}}
\providecommand{\tilmC}{\tilde{\m{C}}}


%##############################################################
\frame{\frametitle{The standard LCA procedure} 
%##############################################################


\benum
\item Define the life phases of the product in question:
\bi
\item production
\item operation/usage
\item destruction/recycling.
\ei

\pause \item For each life phase, calculate the amount of needed
  materials/energy resulting in the \bfdef{life-cycle inventory}
  $\tily_j$ for product category $j$ 
(the tilde denotes that the product is given in physical units such as
  kg or kWh 
  rather than in \euro).

\pause \item The total emissions $e_i$ of pollutant $i$ during the life time is
  obtained using the \bfdef{emission factor matrix} $\m{C}$:

  \maindm{e_i=\sum_j C_{ij}\tily_j}

where the emission factor $C_{ij}$ gives the units of pollutant
$i$ caused by one unit of product $j$ (including the production
chain).

\end{enumerate}

}

%##############################################################
\frame{\frametitle{Example: Gasoline  vehicle}
%##############################################################

Gasoline and Diesel vehicles are two examples of \bfdef{internal
  combustion vehicles (ICV)}

\vspace{1em}

\bfblack{1. Life-cycle inventory}
\vspace{1em}

\bi
\item $\tily_1=\unit[800]{kg}$ steel (\unit[900]{kg} at production
  time, \unit[80]{kg} spare parts during lifetime, \unit[20]{\%}
  emission-neutral recycling contribution),
\item $\tily_2=\unit[60]{kg}$ aluminum (\unit[100]{kg} production,
  \unit[40]{\%} of it can be recycled without additional emissions)
\item $\tily_3=\unit[100]{kg}$ plastic
\item $\tily_4=\unit[50]{kg}$ rubber
\item $\tily_5=\unit[36]{kg}$ lead (three starter batteries \`a \unit[12]{kg})
\item $\tily_6=\unit[15\,000]{l}$ gazoline (\unit[250\,000]{km} at
  \unit[6]{l/\unit[100]{km}} during
  lifetime)
\ei
so we have

\maindm{
\vec{\tily}=(\unit[800]{kg},\ \unit[60]{kg},\ \unit[60]{kg},\
\unit[60]{kg},\ \unit[36]{kg},\ \unit[15\,000]{l})\tr.
}

}

%##############################################################
\frame{\frametitle{Example: Gasoline  vehicle (ctnd)}
%##############################################################

\vspace{1em}
\bfblack{2. Total \COii\ emissions}
\vspace{1ex}

{%\small
Defining $e_1$ to be the \COii\ emissions in kg ($e_2$ could be
NO$_x$, $e_3$ PM and so on), we have
\bdm
e_1=\sum_{j=1}^6 C_{1j}\tily_j
\edm
\vspace{-1em}

with the row vector 
\vspace{-1ex}

\maindm{
\vec{C}\sub{\COii}=(C_{1j})=(4,\ 30,\ 2,\ 2,\ 20,\ \unit[2.7]{kg/l}).
}
The last emission factor $C_{16}=C_{16}\sup{w2t}+C_{16}\sup{t2w}$ is
the sum of two contributions: 

\bi
\pause \item \bfdef{Well-to-tank (w2t)}
emissions of the production chain mining $\to$ transport to
refinery $\to$ refining process $\to$ transport to the gas station:
$C_{16}\sup{w2t}=\unit[0.4]{kg/l}$,\\[1ex]
\pause \item \bfdef{Tank-to-wheel (t2w)} emissions dictated by the chemistry
during the actual combustion: $C_{16}\sup{t2w} = \unit[2.3]{kg/l}$
(it would be \unit[2.7]{kg/l} for Diesel, i.e., the total w2w
emissions of gasoline are about the t2w emissions when burning Diesel).
\ei
}

}

%##############################################################
\frame{\frametitle{Example 2: Battery-electric vehicle (BEV)}
%##############################################################

{\normalsize
\bi
\item The \bfdef{Life-cycle inventory} of steel, aluminum, rubber, plastic
etc is comparable to that of the ICVs.\\[1ex] 
\pause \item The starter batteries are replaced by the Lithium driving
  batteries (2$\times$ \unit[300]{kg}) and the 
  gasoline is replaced 
  by the needed electrical energy, typically \unit[20]{kWh} per
  \unit[100]{km}:

\maindm{
\vec{\tily}=(\unit[800]{kg},\ \unit[60]{kg},\ \unit[60]{kg},\
\unit[60]{kg},\ \unit[600]{kg},\ \unit[50\,000]{kWh})\tr.
}
\vspace{1em}

\pause \item This leads to the new \COii\ emission factors vector
\maindm{
\vec{C}\sub{\COii}=(4,\ 30,\ 2,\ 2,\ 20,\ \unit[0.45]{kg/kWh}).
}
\vspace{1em}

\bi
\item The Li driving batteries are expensive to produce and there is
  much controversy in extimating their overall emission factor
  $C_{15}$\\[1ex]
\pause \item The enery emission factor is based, e.g., on the present (2019)
  German energy mix emitting \unit[450]{g} \COii\ per kWh of electrical
  energy at the socket 
\ei

\ei
}

}

%##############################################################
\frame{\frametitle{Questions on LCA}
%##############################################################

\bi
\itemAsk \colAsk{Is it possible to check, at a glance, whether
  the example BEV emits less \COii\  per km than the example ICV \emph{when
    considering the driving phase alone}?}\\[1ex]

\pause \itemAnswer \colAnswer{Per \unit[100]{km}, the BEV indirectly emits
\unit[20]{kWh} * \unit[0.45]{kg/kWh}=\unit[9]{kg}. The ICV vehicle
emits directly and indirectly
\unit[6]{l} * \unit[2.7]{kg/l}=\unit[16.2]{kg}. So, the BEV ``wins''
when considering the direct and indirect emissions in the driving
phase alone. 
\\[1ex]
However, the BEV production emissions are significantly
higher. Furthermore, less than ideal efficiencies when
charging/discharging have not been considered.}\\[1em]

\pause \itemAsk \colAsk{How would you proceed to calculate the
  \emph{break-even} mileage beyond which a BEV is more environmentally
  friendly (``green'') than the ICV?}\\[1ex]

\pause \itemAnswer \colAnswer{We saw that the \emph{driving} emissions $C'$
  per kilometer $x$ for the ICV are higher compared to the BEV. In
  contrast, it is the other way round for the \emph{fixed} emissions
  $C^0$ due to 
  production/disposal/recycling.
So, just calculate the break-even kilometrage $x_c$ by the equation
\bdm
C^0\sub{BEV}+C'\sub{BEV}x_c= C^0\sub{ICV}+C'\sub{ICVV}x_c
\edm
}
\ei
}

%##############################################################
\frame{\frametitle{Questions on LCA (ctnd.)}
%##############################################################

\bi


\itemAsk \colAsk{Give the two most important factors influencing the total LCA
  emissions of battery-electric vehicles.}

\pause \colAnswer{
\bi
\item[(i)] \colAnswer{The energy mix of the used electricity
  (this is tricky! particularly, you cannot save your soul by paying
  indulgences/ordering ``green'' electricity)}
\pause \item[(ii)] \colAnswer{The production and disposal/recycling emissions of the
  battery and whether you need more than one battery during lifetime
  (to research this is even more trickky).}
\ei
}



\pause \itemAsk \colAsk{A common saying states that \emph{the Sun does not
  issue invoices} nor does the production of electric energy by
  photovoltaic (PV) elements entail
  any direct \COii\ emissions. Discuss why PV energy still has a nonzero
  \COii\ footprint and how to calculate the PV \COii\ emission
  factor. Use LCA arguments and assume a steady state. } 

\pause \colAnswer{
\bi
\item[(i)] \colAnswer{Get information about the usable lifetime $\tau$,}
\pause \item[(ii)] \colAnswer{Check the climate where you want 
to install your PV and
  determine the average power (in Germany, it is about \unit[10]{\%} of the
  installed power $P\sub{max}$) and calculate the total electric
  energy delivered, e.g., $W\sub{el}=0.1\, \tau P\sub{max} $ }
\pause \item[(iii)] \colAnswer{Get the production and recycling 
emissions $C$ of your PV including 
 the connection to the electric grid and calculate the
 \COii\ footprint $e\sub{PV}=C/W\sub{el}$ [kg/kWh].}
\ei
}

\ei
}


\subsection{11.3. EIO-LCA}

%##############################################################
\frame{\frametitle{11.3. Econometric Input-Output LCA}
%##############################################################

See the
\href{https://mtreiber.de/Vkoek_Ma_Skript/Verkehrsoekonometrie_Ma.pdf}{\beamerbutton{German
    script, Chapter 5.3}}.

}


\end{document}
