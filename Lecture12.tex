
\documentclass[mathserif,aspectratio=1610]{beamer}
%\documentclass[mathserif,handout,aspectratio=1610]{beamer}
%\usepackage{beamerthemeshadow}

\input{$HOME/tex/inputs/defsSkript}%$
\input{$HOME/tex/inputs/styleBeamerVkoekMa}%$
%\input{styleBeamerVkoekMa}%$

\usepackage{graphicx}

%\newcommand{\pathDiscrChoice}{$HOME/vorlesungen/Verkehrsoekonometrie_Ma/discrChoice_cc_Levmar} %$
\newcommand{\pathDiscrChoice}{figsDiscr} 

%##############################################################

\begin{document}
%###################################################
\frame{  %title layout
%###################################################
\makePale{1.00}{0.50}{0.70}{1.40}{1.40}

\placebox{0.50}{0.42}{
  \figSimple{1.10\textwidth}{figsIOM/AKWfig.png}}
%makePale{opacity}{centerXrel}{centerYrel}{wrel}\{hrel} 
\makePale{0.80}{0.50}{0.70}{1.40}{1.40}

\placebox{0.55}{0.89}{\parbox{0.65\textwidth}{\myheading{
12 \ Input-Output Model\\ \hspace*{1.3em} of Leontief}}}

\placebox{0.50}{0.14}{\parbox{0.6\textwidth}{
{\large
\bi
\item 12.1. Motivation
\item 12.2. Specification of the Input-Output Model (IOM) of Leontief
\item 12.3. Example: 1=transportation sector, 2=vehicle
  construction
\ei
}
}}

}




\subsection{12.1. Motivation}

%##############################################################
\frame{\frametitle{12.1. Motivation for input-output modelling}
%##############################################################

\vspace{0em}
\fig{0.9\textwidth}{figsIOM/AKWfig.png}
\vspace{-0.5em}

\bi
\item Atomic power plants do not have any direct \COii\ emissions
\pause \item However, what are the \emph{effective} emission considering all
  involved processes recursively?
\ei
}


%##############################################################
\frame{\frametitle{Problem statement}
%##############################################################

\begin{center}
\parbox{0.8\textwidth}{
\bi
\item In a modern economy, nearly everything is connected to ``the
  rest'' of the economy.\\[1ex]
\pause \item \emph{Wanted:} a quantitative description of the flows of
materials, products,
  services, and information between the different parts of an economy.
\pause \item The \bfdef{input-output model (IOM) of Leontief} tackles
this problem by 
  making several assumptions:\\[1ex]
\bi
\item Every material, product, or service is associated with a
  certain \bfdef{sector}\\[1ex]
\pause \item To make all flows (kg, \euro, bytes, ...) commensurable, the
  common unit is a \emph{monetary unit}, e.g., \euro\\[1ex]
\pause \item The whole system is \emph{linear} and \emph{deterministic}: douple input means double output.
  Particularly, there is no economy of scale\\[1ex]
\pause \item The whole system is in the \bfdef{steady state}, e.g., there are no
  temporal changes (constant supply and demand); storage (if
  applicable)  is
  neither built up nor depleted.
\ei
\ei
}
\end{center}

}

\subsection{12.2. Specification of the IOM of Leontief}

%##############################################################
\frame{\frametitle{12.2 Specification of the IOM of Leontief}
%##############################################################

Linear, deterministic coupling of $n$ \bfdef{sectors} and an
\bfdef{end consumer}  in
the steady state: 

\maindm{
\visible<1->{x_i=}
\visible<2->{y_i+}
\visible<3->{\sum\limits_{j=1}^n x_{ij}}
\visible<4->{=y_i+\sum\limits_{j=1}^n A_{ij} x_j}
}

\bi
\item $x_i$: Total output of sector $i$ in \euro{} or other monetary
  units per time unit\\[1ex]
\pause \item $y_i$: Flow of products/services of sector $i$ to the end
  consumers (and to sectors that are not explicitely considered)\\[1ex]
\pause \item $x_{ij}$: Flow from sector $i$ to $j$: Sector $j$ needs a supply
  $x_{ij}$ from sector $i$ to maintain the steady state and to ensure
  a constant 
  supply $y_j$ to the end consumer\\[1ex]
\pause \item $A_{ij}=x_{ij}/x_j$: \bfdef{IO coefficient}
  reflecting linearity: In order to
  produce one unit, sector $j$ needs $A_{ij}$ units from all the other
  sectors $i$, including the own.
\ei


}

%##############################################################
\frame{\frametitle{Visualisation of the flows 
generated by atomic power plants}
%##############################################################
\vspace{-0.5ex}
\fig{0.95\textwidth}{figsIOM/AKWflussdiag_eng.png}
}


%##############################################################
\frame{\frametitle{Total production for a given consumer's supply}
%##############################################################

IOM equation in vector-matrix notation:
\maindm{\vec{x}=\m{A}\cdot \vec{x}+\vec{y}}
\bi
\item $\vec{x}= (x_1,x_2, \cdots, x_n)\tr$ \bfdef{production vector}
\item $\vec{y}= (y_1,y_2, \cdots, y_n)\tr$ \bfdef{supply vector}
\item $\m{A}=(A_{ij}), i,j=1 \cdots n$ \bfdef{IOM coefficient matrix}
\ei
\vspace{1em}
\pause Solving for $\vec{x}$ by writing $(\m{1}-\m{A})\vec{x}=\vec{y}$:
\maindm{\vec{x}=(\m{1}-\m{A})^{-1}\vec{y} \equiv \m{B}\vec{y} }
\bi
\item $\m{B}=(\m{1}-\m{A})^{-1}$ 
\bfdef{coefficient matrix of the final demand}
\ei
}


%##############################################################
\frame{\frametitle{Meaning of the matrix of the final
  demand \m{B}}
%##############################################################

\bi
\item $B_{ij}$ denotes the needed total production from sector $i$ in
  order to deliver one unit of $j$ to the end consumer (or the not
  considered sectors) in the steady state\\[1em]
\pause \item $\m{B}$ includes all indirect effect \emph{in an infinite
  recursion} as can be seen from the Taylor expansion:
\ei
\vspace{1ex}

\bdm
\m{B}=\left(\m{1}-\m{A}\right)^{-1}
= \m{1}+\m{A}+\m{A}^2+\m{A}^3+\cdots
= \sum\limits_{j=0}^{\infty}\m{A}^j
\edm

}

\subsection{12.3. Example}

%##############################################################
\frame{\frametitle{12.3. Example: 1=transportation sector, 2=vehicle
  construction } 
%##############################################################


{\small
\bdm
B_{11}= 1+
\visible<2->{A_{11}+}
\visible<3->{\sum\limits_{k=1}^2 A_{1k}A_{k1}+}
\visible<4->{\sum\limits_{k=1}^2\sum\limits_{l=1}^2A_{1k}A_{kl}A_{l1} + \ \ldots}
\edm

\bi
\item 1: Transportation of the passengers (``end consumers'')
\visible<2->{\item $A_{11}$: 
  The drivers, conductors, and the administrative staff of
the transportation companies need transportation themselves}
\visible<3->{\item $A_{11}^2$: 
  The transport of employees of the transportation companies induces
  additional traffic, hence the need for additional employees to scale
  up the supply accordingly}
\visible<3->{\item $A_{12}A_{21}$: 
To manage operations, the transport sector must offer aditional
transportation for the commutes of the workers/employees of the
vehicle making sector ($A_{12}$), so they can provide additional 
vehicles ($A_{21}$) needed by the transportation sector to maintain the steady
  state. }
\visible<4->{\item $A_{11}A_{12}A_{21}$: 
Since also the employees of
the transportation companies need transportation ($A_{11}$), even more
transportation supply ($A_{12}$) must be offered to the
employees of the vehicle making companies to get the additionally 
needed vehicles ($A_{21}$)
\item $\cdots$}
\ei
}


}

%##############################################################
\frame{\frametitle{Questions} 
%##############################################################

\bi
\itemAsk \colAsk{Argue that a national economy with sectors $i$ satisfying 
$\sum_jA_{ij}x_j>x_i$ would not be sustainable or needs external help
  (``GDR'').}

\pause\itemAnswer\colAnswer{In such an economy, sector $i$  
must deliver more units to operate itself ($A_{ii}x_i$) and the
 other sectors 
 ($A_{ij}x_j$) than this sector produces in total ($x_i$).}

\pause\itemAsk \colAsk{Give reasons why all $A_{ij}$ and $B_{ij}$ are $\ge 0$ and
  $B_{ii}\ge 1$.}
\pause\itemAnswer\colAnswer{Since sectors \emph{need} products and
services from other sectors. }

\pause\itemAsk \colAsk{Assume that the external demand $y_k$ for
  products/services of 
  sector $k$ suddenly increases by $r_k=\unit[1]{\%}$ (e.g., driven by
  politics). Give a 
  general expression for the
  percentaged increase of the GDP in
  order to re-attain the steady state.}

\pause\itemAnswer\colAnswer{The change of the demand vector is
    given by $\Delta \vec{y}=(0,..,r_ky_k,0,...)\tr$ and the change of
the production vector components by
$\Delta x_i=\sum_jB_{ij}y_j=r_kB_{ik}y_k$. Hence, the change of the 
total GDP is given by $\Delta x=\sum_i\Delta x_i=r_k\sum_i B_{ik}y_k$
and the old GDP itself by
$x=\sum_ix_i=\sum_i\sum_jB_{ij}y_j$. Finally, the
percentage increase of the total GDP  is given by $\Delta x/x$}
\ei


}




%##############################################################
\frame{\frametitle{Questions (ctnd.)} 
%##############################################################

\bi
\itemAsk \colAsk{Give some additional elements and concepts
  needed to make the 
  IOM  dynamic}
%\ei}\end{document}

\pause\itemAnswer\colAnswer{
After a sudden change of the demand,
  the demand vector $\vec{y}$ is no longer balanced against the
  available production $(\m{1}-\m{A})\vec{x}$ and the excess demand or
  supply is balanced by emptying or filling the stores. If the economy is
  \bfdef{demand-driven (Keynes)}, this also induces ramping up/down
  the production. In the simplest case, the rate of change of the
  production is proportional to the excess demand,
\bdm
\abl{x_i}{t}
=\frac{1}{\tau_i}\bigg[y_i(t)-\big( (\m{1}-\m{A})\vec{x}\big)_i\bigg] 
\edm
where $\tau_i$ is the time the sector $i$ needs to adapt to changing
demands.
}
\\[1ex]

\pause\itemAsk \colAsk{Give some additional elements and concepts
  needed to introduce nonlinearity reflecting the economy of scale}

\pause\itemAnswer\colAnswer{
In an \bfdef{economy of scale}, the IO coefficients become
  smaller with the number of produced units of the target sector which
  may be modelled, 
  e.g., by
\bdm
A_{ij}(x_j)=\frac{A_{ij}(0)}{1+x_j/x_{j0}}
\edm
where $x_{j0}$ is the production quantity where significal scale
effects set in.
}
\ei

}


\end{document}
