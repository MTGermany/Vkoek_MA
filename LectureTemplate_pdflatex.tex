


%#############################################
% pdflatex beamer template
%
% only png, jpg (or svg?) images but positioning now ok
% authoritative sources
%~/tex/inputs/LectureTemplate_pdflatex.tex
%~/tex/inputs/LectureTemplate_latex.tex
%#############################################


\documentclass[mathserif,aspectratio=1610]{beamer}
%\documentclass[mathserif,handout,aspectratio=1610]{beamer}
%\usepackage{beamerthemeshadow}
\input{$HOME/tex/inputs/defsSkript}  %$
%\input{./defsSkript}  %$
\input{$HOME/tex/inputs/styleBeamerPdflatex}   %$


\usepackage{graphicx}
%\setbeamercovered{invisible}   % no shining through of future layers
%\setbeamercovered{transparent} % constant transparency of all future layers
                                % default: progressive shining through


%##############################################################

\begin{document}

%########################################################
\frame{\frametitle{Scrolling possible in fullscreen mode?}
%########################################################

see scrolltest.tex; although complete, does not work on ubuntu12
}


\section{10. Title of this lecture}


%##############################################################
\frame{\frametitle{\hypertarget{general}{} General and Bugs}
%##############################################################

\bi
\item use with png,jpg images with styleBeamerPdflatex and pdflatex;
  placebox works 
\item use with eps images with png changed to eps in
  styleBeamerPdflatex (->would be styleBeamerLatex) and tex2pdf
\item Local link targets placed at first occurrence of page
\item neither placebox nor makePale works for latex version
\itemAnswer \myLocalLink{sec:defs}{Link zu Target sec:defs, 
angelegt mit hypertarget}
\ei

}

%##############################################################
\frame{\frametitle{Placebox test}
%##############################################################

\placebox{0.50}{0.49}
 {\figSimple{0.80\textwidth}{figsAllg_vpl/Vierstufenmodell.png}}

\makePale{0.80}{0.50}{0.49}{0.90}{0.75}

\placebox{0.50}{0.82}{\parbox{0.95\textwidth}{
 \myheading{10. Title\\of this lecture}} %also mysubheading
} 

\placebox{0.50}{0.45}
 {\parbox{0.8\textwidth}{
  \bi
  \item 10.1 first
  \item 10.2 second
  \ei
}}


}

%###############################################################
\frame{\frametitle{pause with images}
%###############################################################
\bi
\item Images per default not ``paused'' because of some bugs when
  attempting making them transparent
\pause \item maybe setbeamercovered\{invisible\} works (with side
effect future elements not transparent)
\pause \item the visible command seems to override this (making future imgs
opaque as in setbeamercovered\{invisible\} but only there:
\visible<3->{

 \fig{0.4\textwidth}{$HOME/tex/inputs/TUlogo_weissAufBlau.png}%$
% figSimple{0.4\textwidth}{TUlogo_weissAufBlau.png}%$
}
\pause item the strange argument 3- indicates image shown after 3rd
click as any text after the second ``pause'' keyword

\pause with placebox: \texttt{visible<3->\{ placebox\{ ...\} \}}
\ei
}

%###############################################################
\frame{\frametitle{pause within tabular}
%###############################################################
\begin{tabular}{|r|r|} \hline
a & b\\ \hline
\visible<2->{c & d}\\ \hline
\visible<2->{e & f}\\ \hline
\visible<3->{g & h}\\ \hline
\end{tabular}

\bi
\item Example uncovering whole rows at a time (pause uncovers elements)
\pause \item uncover several elements simultaneously $\Rightarrow$ same
  number in \texttt{visible}
\pause \item \texttt{visible<2->\{...\}} remains visible for rest of slide
\pause \item \texttt{visible<2>\{...\}} remains visible only at second
  click=second pause
\item uncover a whole row
\ei
}

\subsection{10.1 first} 

%###############################################################
\frame{\frametitle{visible instead of pause to temporarily override
    semitransparency of future kayers}
%###############################################################
  \bi
  \item 6.1 Allgemeines zur Umlegung
  \visible<2->{\item 6.2 Nachfrageseite: Fahrtenmatrizen}
  \visible<3->{\item 6.3 Netzmodellierung}
  \visible<4->{\item 6.4 Capacity-Restraint Funktionen}
  \visible<5->{\item 6.5 Erstes Wardrop'sches Prinzip: Nutzergleichgewicht}
  \visible<6->{\item 6.6 Zweites Wardrop'sches Prinzip: Systemoptimum}
  \visible<7->{\item 6.7 Zusammenhang zwischen Nutzergleichgewicht und
    Systemoptimum} 
  \visible<8->{\item 6.8 Fun Fact: Das Braess'sche Paradoxon}
  \ei
}

%###############################################################
\frame{\frametitle{visible braces can jump environemnts!}
%###############################################################

\bdm
\begin{array}{lll}
T_1(q,w_1)  &=& 10 + 8 q w_1 + q,\\
T_2(q,w_1)  &=&  8 - 3 q w_1 + 4 q
\end{array}
\edm

\visible<2->{
\bdma 
T\sub{sys}(\vec{w}) &=& w_1T_1(q,w_1)+w_2T_2(q,w_1)\\
 &=& w_1T_1(q,w_1)+(1-w_1)T_2(q,w_1) \\[-0.3em]
 &=& 11 q w_1^2+(2-6q)w_1 + 8+4q \stackrel{!}{=}\min_{w_1}\\[-0.3em]}
\visible<3->{T'\sub{sys}(w_1) &=& 22 q w_1+2-6q\stackrel{!}{=}0  \Rightarrow \\[-0.3em]}
\visible<4->{w_1\sup{SO}
 &=& \twoCases{\frac{3q-1}{11q}}{\ q\le \frac{1}{3}}{0}{
\ \text{sonst.}}}
\edma
%Die Fallunterscheidung ist durch die Bedingung $w_1\sup{SO}\ge 0$ verursacht
}

%###############################################################
\frame{\frametitle{10.1 first}
%###############################################################

The \emph{plausibility criteria} of the last lesson and model
completeness are necessary but
not sufficient for a realistic simulation. Additional requirements include
\bi
\item No accidents (CF models are expected to reflect \emph{regular} 
behaviour \red{$\Rightarrow$ not satisfied by the OVM and FVDM})

\item The accelerations $\dot{v}$ and braking decelerations have to be
  physically possible, e.g. $-\unit[9]{m/s^2}\le \dot{v} \le
  \unit[4]{m/s^2}$  \red{$\Rightarrow$ not satisfied by the OVM, Newell's
    micromodel or the CA models}
\item Furthermore, they have to be comfortable in normal situations,
  e.g., $|\dot{v}|<\unit[2]{m/s^2}$ depending on the driving style
  \red{$\Rightarrow$ distinguish between emergency and normal driving}
\item For highly dynamic situations such as approaching to red traffic
  lights/standing vehicles, elementary kinematics, e.g., the minimum
  stopping deceleration $b\sub{kin}=v^2/(2s)$ has to be
  contained \red{$\Rightarrow$ incorporate some driving strategy}
\item The model parameters should reflect aspects of the driving style
\ei
}




\subsection{10.2 second}


%########################################################
\frame{\frametitle{\hypertarget{sec:defs}{} 10.2 useful definitions}
%########################################################

\vspace{2em}

myheading, mysubheadingq, mysubsubheading, bfsf, 

maineq\{eq label\}\{formula\},
maindm\{eq\}, maindmIntext\{eq\}, maintextbox\{width\}\{text\},
maintextbox\{text\}, fboxdm, fboxtext, myBox

colDef, colAsk, colAnswer, bfAsk,
bfAnswer,

itemAsk, itemAnswer, itemGray,

makePale\{opacity\}\{centerXrel\}\{centerYrel\}\{wrel\}\{hrel\} to
make images pale, circled\{char\}



}




\end{document}




















