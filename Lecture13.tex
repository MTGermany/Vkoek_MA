
\documentclass[mathserif,aspectratio=1610]{beamer}
%\documentclass[mathserif,handout,aspectratio=1610]{beamer}
%\usepackage{beamerthemeshadow}

\input{$HOME/tex/inputs/defsSkript}%$
\input{$HOME/tex/inputs/styleBeamerVkoekMa}%$
%\input{styleBeamerVkoekMa}%$

\usepackage{graphicx}

%\newcommand{\pathDiscrChoice}{$HOME/vorlesungen/Verkehrsoekonometrie_Ma/discrChoice_cc_Levmar} %$
\newcommand{\pathDiscrChoice}{figsDiscr} 

%##############################################################

\begin{document}
%###################################################
\frame{  %title layout
%###################################################
\makePale{1.00}{0.50}{0.70}{1.40}{1.40}

\placebox{0.50}{0.42}{
  \figSimple{1.60\textwidth}{figsIOM/AKWfig.png}}
%makePale{opacity}{centerXrel}{centerYrel}{wrel}\{hrel} 
\makePale{0.80}{0.50}{0.70}{1.40}{1.40}

\placebox{0.55}{0.89}{\parbox{0.45\textwidth}{\myheading{
13 \ Life-Cycle Assessment}}}

\placebox{0.43}{0.14}{\parbox{0.8\textwidth}{
{\large
\bi
\item 13.1. Classical Life-Cycle Assessment (LCA)
\item 13.2. Econometric Input-Output LCA (EIO-LCA)

\ei
}
}}

}



\subsection{13.1. Classical Life-Cycle Assessment}

%##############################################################
\frame{\frametitle{13.1. Life-Cycle Assessment (LCA): Motivation} 
%##############################################################

\bi
\item The IOM reflects a \emph{snapshot} of \emph{all products} of a
  national economy in the \emph{steady  state}

\pause \item Sometimes, it is more instructive or relevant to consider
  \emph{the total 
  lifetime} of \emph{a single product} in a \emph{time dependent way}
  by assessing production, operation, and destruction/recycling of
  this product. 

\pause \item This is formalized by the methods of \bfdef{Life-Cycle
  Assessment (LCA)} (German: \bfdef{\"Okobilanz}).

\pause \item However, LCA only considers first-order indirect effects, e.g.,
  \COii\ emissions caused by electric vehicles through the
  \COii\ footprint of electricity production

\pause \item The class of \bfdef{Econometric Input-Output (EIO)
  LCA models} combines both approaches. 
\ei
}



\providecommand{\tily}{\tilde{y}}
\providecommand{\tilC}{\tilde{C}}
\providecommand{\tilmC}{\tilde{\m{C}}}


%##############################################################
\frame{\frametitle{The standard LCA procedure} 
%##############################################################


\benum
\item Define the life phases of the product in question:
\bi
\item production
\item operation/usage
\item destruction/recycling.
\ei

\pause \item For each life phase, calculate the amount of needed
  materials/energy resulting in the \bfdef{life-cycle inventory}
  $\tily_j$ for product category $j$ 
(the tilde denotes that the product is given in physical units such as
  kg or kWh 
  rather than in \euro).

\pause \item The total emissions $e_i$ of pollutant $i$ during the life time is
  obtained using the \bfdef{emission factor matrix} $\m{C}$:

  \maindm{e_i=\sum_j C_{ij}\tily_j}

where the emission factor $C_{ij}$ gives the units of pollutant
$i$ caused by one unit of product $j$ (including the production
chain).

\end{enumerate}

}

%##############################################################
\frame{\frametitle{Example: Gasoline  vehicle}
%##############################################################

Gasoline and Diesel vehicles are two examples of \bfdef{internal
  combustion vehicles (ICV)}

\vspace{1em}

\bfblack{1. Life-cycle inventory}
\vspace{1em}

\bi
\item $\tily_1=\unit[800]{kg}$ steel (\unit[900]{kg} at production
  time, \unit[80]{kg} spare parts during lifetime, \unit[20]{\%}
  emission-neutral recycling contribution),
\item $\tily_2=\unit[60]{kg}$ aluminum (\unit[100]{kg} production,
  \unit[40]{\%} of it can be recycled without additional emissions)
\item $\tily_3=\unit[100]{kg}$ plastic
\item $\tily_4=\unit[50]{kg}$ rubber
\item $\tily_5=\unit[36]{kg}$ lead (three starter batteries \`a \unit[12]{kg})
\item $\tily_6=\unit[15\,000]{l}$ gazoline (\unit[250\,000]{km} at
  \unit[6]{l/\unit[100]{km}} during
  lifetime)
\ei
so we have

\maindm{
\vec{\tily}=(\unit[800]{kg},\ \unit[60]{kg},\ \unit[60]{kg},\
\unit[60]{kg},\ \unit[36]{kg},\ \unit[15\,000]{l})\tr.
}

}

%##############################################################
\frame{\frametitle{Example: Gasoline  vehicle (ctnd)}
%##############################################################

\vspace{1em}
\bfblack{2. Total \COii\ emissions}
\vspace{1ex}

{%\small
Defining $e_1$ to be the \COii\ emissions in kg ($e_2$ could be
NO$_x$, $e_3$ PM and so on), we have
\bdm
e_1=\sum_{j=1}^6 C_{1j}\tily_j
\edm
\vspace{-1em}

with the row vector 
\vspace{-1ex}

\maindm{
\vec{C}\sub{\COii}=(C_{1j})=(4,\ 30,\ 2,\ 2,\ 20,\ \unit[2.7]{kg/l}).
}
The last emission factor $C_{16}=C_{16}\sup{w2t}+C_{16}\sup{t2w}$ is
the sum of two contributions: 

\bi
\pause \item \bfdef{Well-to-tank (w2t)}
emissions of the production chain mining $\to$ transport to
refinery $\to$ refining process $\to$ transport to the gas station:
$C_{16}\sup{w2t}=\unit[0.4]{kg/l}$,\\[1ex]
\pause \item \bfdef{Tank-to-wheel (t2w)} emissions dictated by the chemistry
during the actual combustion: $C_{16}\sup{t2w} = \unit[2.3]{kg/l}$
(it would be \unit[2.7]{kg/l} for Diesel, i.e., the total w2w
emissions of gasoline are about the t2w emissions when burning Diesel).
\ei
}

}

%##############################################################
\frame{\frametitle{Example 2: Battery-electric vehicle (BEV)}
%##############################################################

{\normalsize
\bi
\item The \bfdef{Life-cycle inventory} of steel, aluminum, rubber, plastic
etc is comparable to that of the ICVs.\\[1ex] 
\pause \item The starter batteries are replaced by the Lithium driving
  batteries (2$\times$ \unit[300]{kg}) and the 
  gasoline is replaced 
  by the needed electrical energy, typically \unit[20]{kWh} per
  \unit[100]{km}:

\maindm{
\vec{\tily}=(\unit[800]{kg},\ \unit[60]{kg},\ \unit[60]{kg},\
\unit[60]{kg},\ \unit[600]{kg},\ \unit[50\,000]{kWh})\tr.
}
\vspace{1em}

\pause \item This leads to the new \COii\ emission factors vector
\maindm{
\vec{C}\sub{\COii}=(4,\ 30,\ 2,\ 2,\ 20,\ \unit[0.45]{kg/kWh}).
}
\vspace{1em}

\bi
\item The Li driving batteries are expensive to produce and there is
  much controversy in extimating their overall emission factor
  $C_{15}$\\[1ex]
\pause \item The enery emission factor is based, e.g., on the present (2019)
  German energy mix emitting \unit[450]{g} \COii\ per kWh of electrical
  energy at the socket 
\ei

\ei
}

}

%##############################################################
\frame{\frametitle{Questions on LCA}
%##############################################################

\bi
\itemAsk \colAsk{Is it possible to check, at a glance, whether
  the example BEV emits less \COii\  per km than the example ICV \emph{when
    considering the driving phase alone}?}\\[1ex]

\pause \itemAnswer \colAnswer{Per \unit[100]{km}, the BEV indirectly emits
\unit[20]{kWh} * \unit[0.45]{kg/kWh}=\unit[9]{kg}. The ICV vehicle
emits directly and indirectly
\unit[6]{l} * \unit[2.7]{kg/l}=\unit[16.2]{kg}. So, the BEV ``wins''
when considering the direct and indirect emissions in the driving
phase alone. 
\\[1ex]
However, the BEV production emissions are significantly
higher. Furthermore, less than ideal efficiencies when
charging/discharging have not been considered.}\\[1em]

\pause \itemAsk \colAsk{How would you proceed to calculate the
  \emph{break-even} mileage beyond which a BEV is more environmentally
  friendly (``green'') than the ICV?}\\[1ex]

\pause \itemAnswer \colAnswer{We saw that the \emph{driving} emissions $C'$
  per kilometer $x$ for the ICV are higher compared to the BEV. In
  contrast, it is the other way round for the \emph{fixed} emissions
  $C^0$ due to 
  production/disposal/recycling.
So, just calculate the break-even kilometrage $x_c$ by the equation
\bdm
C^0\sub{BEV}+C'\sub{BEV}x_c= C^0\sub{ICV}+C'\sub{ICVV}x_c
\edm
}
\ei
}

%##############################################################
\frame{\frametitle{Questions on LCA (ctnd.)}
%##############################################################

\bi


\itemAsk \colAsk{Give the two most important factors influencing the total LCA
  emissions of battery-electric vehicles.}

\pause \colAnswer{
\bi
\item[(i)] \colAnswer{The energy mix of the used electricity
  (this is tricky! particularly, you cannot save your soul by paying
  indulgences/ordering ``green'' electricity)}
\pause \item[(ii)] \colAnswer{The production and disposal/recycling emissions of the
  battery and whether you need more than one battery during lifetime
  (to research this is even more trickky).}
\ei
}



\pause \itemAsk \colAsk{A common saying states that \emph{the Sun does not
  issue invoices} nor does the production of electric energy by
  photovoltaic (PV) elements entail
  any direct \COii\ emissions. Discuss why PV energy still has a nonzero
  \COii\ footprint and how to calculate the PV \COii\ emission
  factor. Use LCA arguments and assume a steady state. } 

\pause \colAnswer{
\bi
\item[(i)] \colAnswer{Get information about the usable lifetime $\tau$,}
\pause \item[(ii)] \colAnswer{Check the climate where you want 
to install your PV and
  determine the average power (in Germany, it is about \unit[10]{\%} of the
  installed power $P\sub{max}$) and calculate the total electric
  energy delivered, e.g., $W\sub{el}=0.1\, \tau P\sub{max} $ }
\pause \item[(iii)] \colAnswer{Get the production and recycling 
emissions $C$ of your PV including 
 the connection to the electric grid and calculate the
 \COii\ footprint $e\sub{PV}=C/W\sub{el}$ [kg/kWh].}
\ei
}

\ei
}


\subsection{13.2. Econometric Input-Output Life-Cycle Assessment (EIO-LCA}

%##############################################################
\frame{\frametitle{13.2. Econometric Input-Output LCA}
%##############################################################

See the
\href{https://mtreiber.de/Vkoek_Ma_Skript/Verkehrsoekonometrie_Ma.pdf}{\beamerbutton{German
    script, Chapter 5.3}}.

}


\end{document}
