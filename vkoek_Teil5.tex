


%#######################################
\chapter{\label{sec:IOM}\label{sec:verfl}\"Okonometrische
  Verflechtungsmodelle}
%#######################################

%\hyphenation{\"Oko-no-me-trie}  FUCK geht nicht mit Umlauten -
%                                 de-facto useless
%\hyphenation{\"Oko-bi-lanz}
%\hyphenation{be-n\"o-tigt}
\hyphenation{Res-sour-cen-ver-brauch}
\hyphenation{Ver-kehrs-be-trie-be}
\hyphenation{Ver-kehrs-leis-tung}
\hyphenation{Ver-kehrs-mit-tel}
\hyphenation{Ver-kehrs-nach-frage}
\hyphenation{Ver-kehrs-sek-tor}
\hyphenation{wirt-schaft-li-cher}

%#######################################
\section{Allgemeines}
%#######################################

Eine wichtige Aufgabenstellung der \"Okonometrie und insbesondere der
Verkehrs\"okono\-me\-trie ist die quantitative Beschreibung der
Str\"ome von Waren, Produkten und sonstiger wirtschaftlicher
Leistungen vom Anbieter (Output) zum Empf\"anger (Input).
Dies ist nichttrivial, da 
Produkte und Dienstleistungen
nicht nur vom Endverbraucher konsumiert werden, sondern auch von den
Herstellern/Anbietern  dieser oder anderer Produkte und Dienstleistungen. 


%#######################################
\begin{figure}
\fig{1.0\textwidth}{figsIOM/AKWfig.eps}
\caption{\label{fig:AKW}
%#######################################
Atomkraftwerke (AKW) emittieren bei der eigentlichen
Stromerzeugung durch Kernspaltung keinerlei CO$_2$. Aber zum Betrieb
eines AKW sind andere Produkte und Dienstleistungen n\"otig, deren
Bereitstellung  sehr
wohl mit der Emission von CO$_2$ verbunden ist: Rohstoffgewinnung, die
Aufbereitung des Rohstoffes zum Brennstoff, Entsorgung, der Transport
von Roh- Brenn- und Abfallstoffen sowie der elektrischen Energie, Bereitstellung und Betrieb der
Infrastruktur zum Transport von Warenstr\"omen und elektrischer
Energie (Hochspannungsleitung), aber auch anteilig der Bau sowie der Betrieb (u.a. Heizung)
der eigentlichen Bauwerke der Anlage usw. Diese 
Kette muss man, streng genommen, ad infinitum fortsetzen: Unter
anderem sollten auch die Emissionen durch die Heizung der
Verwaltungsgeb\"aude der Firmen, welche die Anlage herstellen oder den
Stahl f\"ur die Strommasten liefern, anteilig enthalten sein.
}
\end{figure}
%#######################################

Ein aktuelles
Beispiel ist die Frage, wie viel CO$_2$ die Atomkraftwerke (AKW) effektiv
zur Herstellung einer kWh an Strom erzeugen (Abb. \ref{fig:AKW} und \ref{fig:AKWflussdiag}).
Die Frage ist nun: Wie hoch ist die  tats\"achliche Kohlendioxidbilanz
(g CO$_2$ pro kWh), wenn man nicht nur den direkten Output, sondern
auch die unendliche Kette all dieser indirekten Emissionen
ber\"ucksichtigt? Wie hoch ist diese Bilanz im Vergleich der Gesamtbilanz f\"ur 
Kohle- oder Gaskraftwerken oder bei der Erzeugung erneuerbarer
Energie?\footnote{Die Durchf\"uhrung der Rechnung ergibt f\"ur AKWs
etwa \unit[40]{g} CO$_2$/kWh, f\"ur Kohlekraftwerke etwa 
\unit[800]{g} CO$_2$/kWh und
f\"ur alternative Energiequellen \unit[50-100]{g} CO$_2$/kWh.}

Generell bewirken diese Kopplungen, dass Nachfrageschwankungen
bez\"uglich eines Wirtschaftsbereiches auch die anderen Sektoren
beeinflussen. Beispielsweise bewirkt eine steigende Nachfrage nach
Kraftfahrzeugen auch eine Konjunktur bei den Herstellern von Stahl,
beim Maschinenbau und bei der Kfz-Elektronik. Dadurch ergibt sich
folgende  Fragestellung:

\maintext{\bfblack{Problemformulierung}:
Wie \"andern sich die Nachfragen nach Produkten oder
Dienstleistungen der verschiedenen Wirtschaftszweige, wenn sich die
Nachfrage der Endverbraucher an Leistungen eines bestimmten Wirtschaftszweiges um
einen bestimmten Betrag \"andert? 
}



%#######################################
\section{\label{sec:IOMdef}Formulierung des Verflechtungsmodells}
%#######################################

Bemerkenswerterweise lassen sich im linearen statischen Grenzfall all
diese Verflechtungen und Kopplungen sehr kompakt durch 
das \"okonometrische \bfdef{Verflechtungsmodell}, auch
\bfdef{Input-Output-Modell (IOM)} oder nach seinem Begr\"under
Wassily Leontief \bfdef{Leontief-Modell}
genannt, beschreiben. Im Rahmen dieses Modells gliedert man die gesamte
Volkswirtschaft in verschiedene Sektoren $i =1, \ldots, n$ auf und
modelliert die Waren- und Dienstleistungsstr\"ome $x_{ij}$ vom
Sektor $i$ nach $j$ sowie die Str\"ome $y_i$ vom Sektor $i$ zum
Endverbraucher\footnote{Der Begriff des ``Endverbrauchers'' kann auch allgemeiner gefasst werden und neben dem
eigentlichen Konsumenten auch alle externen Sektoren umfassen, die man in einer
gegebenen Untersuchung nicht separat erfassen will.}
 durch eine \textit{lineare,
zeitunabh\"angige, deterministische}  Kopplung:
\maineq{IOM}{
x_i=y_i+\sum\limits_{j=1}^n x_{ij} =y_i+\sum\limits_{j=1}^n A_{ij} x_j.
}
Hierbei bedeuten (vgl. auch Abb.~\ref{fig:AKWflussdiag}):
\bi
\item $x_i$ der Gesamtaussto\3 an Produkten oder Leistungen im Sektor
$i$ in mit den anderen Sektoren kommensurablen Einheiten (im Allgemeinen Geldeinheiten),
\item $x_{ij}$ die Menge an Produkten oder Dienstleistungen
des Sektors $i$, welche f\"ur Sektor
$j$ be\-n\"o\-tigt wird,
\item $y_i$ die direkt von Sektor $i$ an den Endverbraucher bzw. an
nicht direkt ber\"ucksichtigte Sektoren gehenden
Produkte oder Leistungen,
\item $A_{ij}$ die Elemente  der  \bfdef{Koeffizientenmatrix des direkten
Aufwandes}, also die Zahl der Einheiten des Produkts/der Dienstleistung $i$, welche
  zur Herstellung/Erbringung einer Einheit des Produkts/der
  Dienstleistung des Sektors $j$ ben\"o\-tigt wird.
\ei
%
Die verrschiedenen Sektoren werden beispielsweise duch
die Input-Output-Rechnung des Bundes definiert. In der aktuellsten Ausgabe 2009 werden
73 Sektoren unterschieden\footnote{Statistisches Bundesamt (2013):
  Volkswirtschaftliche Gesamtrechnungen -- Input-
Output-Rechnung 2009, \url{https://www.destatis.de/} (Zugriff 06.08.2013).}
Das IOM ist durch folgende Annahmen charakterisiert:
\bi
\item Alle G\"uter- und Dienstlestungsstr\"ome  m\"ussen sich \emph{genau
  einem Sektor} zuordnen lassen, z.B. einem der 73 Sektoren der
  Input-Output-Rechnung des Bundes.
\item Alle Str\"ome werden in \emph{Geldeinheiten} pro Zeiteinheit gemessen.
\item Es wird \emph{Stationarit\"at} angenommen, d.h. alle Str\"ome werden
  aus der laufenden Produktion gespeist. Lagerbest\"ande sowie ggf. die
  Altersstruktur der G\"uter (z.B. beim Fahrzeugsektor) bleiben
  unver\"andert.
\item Es wird \emph{Linearit\"at} angenommen. Skaleneffekte,
  beispielsweise Effizienzsteigerungen durch Massenproduktion, bleiben unber\"ucksichtigt.
\ei
%
Im Rahmen der allgemeinen Klassifikation \"okonometrischer Modelle
stellt das Verflechtungsmodell ein lineares, statisches,
deterministisches Modell mit Kopplung dar:


%#######################################
\begin{figure}
\fig{0.9\textwidth}{figsIOM/AKWflussdiag.eps}
\caption{\label{fig:AKWflussdiag}Auswahl an G\"uterstr\"omen
 im AKW-Beispiel.}
\end{figure}
%#######################################


\bi
\item Die endogenen Variablen\footnote{Um Kompatibilit\"at mit den
sonstigen Darstellungen des IOM 
zu bewahren, bedeuten hier ausnahmsweise $y_i$
die exogenen und $x_i$ die endogenen Variablen. Ohnehin ist die
Interpretation ``exogen'' bzw. ``endogen'' hier nicht eindeutig zu
treffen. Handelt es sich um eine rein nachfragegetriebene
(Keynesianische) Wirtschaft, ist der Nachfragevektor $\vec{y}$ exogen (die Nachfrage bestimmt
das Angebot $\vec{x}$). Bei angebotsgetriebenen M\"arkten (``der Kunde
muss essen, was auf den Tisch kommt'') ist es umgekehrt.}
sind durch die Gesamtproduktionen $x_i$
der Sektoren gegeben. Die verschiedenen $x_i$ sind gekoppelt, da
 sie auch auf der rechten Seite der Gleichungen vorkommen.
\item Die exogenen Variablen sind durch die externen Nachfragen $y_i$
 gegeben.
\item Die Modellparameter sind durch die Elemente $A_{ij}$ 
gegeben. 
\ei
Je nach Sachverhalt ist es sinnvoll, die $x_i$ und $y_i$ entweder als
\bfdef{Bestandsmassen} oder \bfdef{Bewegungsmassen} zu definieren.
\bi
\item Bewegungsmassen sind nur \"uber Zeit\emph{intervalle} definiert und
nehmen im \bfdef{Flie\3gleichgewicht} linear mit der L\"ange des
Intervalls zu. Es handelt sich um G\"uter\-\emph{men\-gen}, welche
z.B. durch den Wert in \euro{} angegeben werden.
\item Bestandsmassen sind f\"ur Zeit\emph{punkte} definiert. Es handelt sich um G\"uter\emph{str\"ome}, welche
z.B. durch die ``Stromst\"arke'' in \euro{}/h gemessen werden. Im
Flie\3gleichgewicht sind sie konstant.
\ei
H\"aufig ist es am anschaulichsten, alles als Bestandsmassen bei
festen Betrachtungs-Zeitintervall anzusehen, vgl. die Beispiele.

Die Modellparameter, also die \bfdef{Koeffizienten $A_{ij}$  des
direkten Aufwandes}, werden an den \"okonomischen Sachverhalt
angepasst,
 indem die tats\"achlichen Verh\"altnisse in die
Definitionsgleichung

\maineq{Aij}{A_{ij}=\frac{x_{ij}}{x_j}
}
%
eingesetzt werden bzw. das Modell bez\"uglich dieser Parameter
kalibriert wird. In Worten bedeutet Gl. \refkl{Aij}:
\maintext{Der direkte Aufwandskoeffizient 
$A_{ij}$ ist die Zahl der Einheiten des Wirtschaftsgut $i$, welche direkt zur
Herstellung einer Einheit des Produktes $j$ bzw. Erbringung einer
Einheit der Dienstelistung $j$ ben\"otigt wird.}



\verstaendnisbox{Machen Sie sich klar, dass im Falle der Verwendung
kommensurabler Einheiten (z.B. der Waren- oder Dienstleistungswert in
\euro) die Koeffizient des direkten Aufwandes nur im Bereich $0\le
A_{ij}<1$ liegen d\"urfen.}

{\scriptsize \textit{Antwort:} Da sonst nach Gl.~\refkl{IOM} ein Teil des
  G\"uterstroms gr\"o\3er ist als der gesamte Strom.}

%#######################################
\subsection{Kompakte Formulierung in Vektor-Matrix-Notation}
%#######################################

%\renewcommand{\vec}[1]{\underline{#1}}
%\newcommand{\mat}[1]{\underline{\underline{#1}}}

Das Verflechtungsmodell \ref{IOM} stellt ein lineares inhomogenes Gleichungssystem
f\"ur die unbekannten Gesamtproduktionen $x_i$ bei gegebenen externen
Nachfragen $y_i$ dar. Zur Darstellung der formalen (und auch der
expliziten) L\"osung dieses Systems werden die Produktionen und
externen Nachfragen durch Spaltenvektoren dargestellt:
\bdma
\vec{x}^T &=& (x_1,x_2, \ldots, x_n),\\
\vec{y}^T &=& (y_1,y_2, \ldots, y_n).
\edma
Hierbei bedeutet, wie im Kapitel~\ref{sec:regr},
  das Superskript T, ``transponiert''.
Damit kann man \refkl{IOM} schreiben als

\be
\label{IOMMatrixEx}
\myVector{x_1 \\ \vdots \\ x_n} 
 = \myMatrixThree{A_{11}&\cdots & A_{1n}\\ 
                              \vdots && \vdots \\
                            A_{n1}&\cdots & A_{nn}
} \cdot  \myVector{x_1 \\ \vdots \\ x_n}
+ \myVector{y_1 \\ \vdots \\ y_n}
\ee
oder
\maineq{IOMMatrix}{
\vec{x}=\m{A}\cdot \vec{x}+\vec{y}.
}

%#######################################
\section{L\"osung des Verflechtungsmodells}
%#######################################

Mit der Einheitsmatrix 
\bdm
\m{1}=\myMatrixFour{1 & 0 & \cdots & \\
 0 & \ddots &  & \vdots \\
  \vdots & & \ddots & 0\\
  & \cdots & 0 & 1}
 \edm
und
\bdm
\m{M}=\m{1}-\m{A}
\edm
kann man Gl. \refkl{IOMMatrix} schreiben als
\bdm
\m{M}\cdot \vec{x}=\vec{y}.
\edm
Ist $\m{M}$ invertierbar, gibt es also eine Inverse $\m{M}^{-1}$
mit $\m{M}^{-1}\cdot \m{M}=\m{M}\cdot \m{M}^{-1}=\m{1}$, so
kann man diese Gleichung durch Links-Multiplikation von $\m{M}^{-1}$
``l\"osen'':
\bdma
\m{M}^{-1}\cdot \m{M}\cdot \vec{x} &=& \m{M}^{-1}\cdot \vec{y},\\
 \vec{x} &=& \m{M}^{-1}\cdot \vec{y}.
\edma
Dies f\"uhrt nach Einsetzen der Definition von $\m{M}$ zur formalen
L\"osung des Verflechtungsmodells:
\maineq{IOMloes}{
 \vec{x}=\left(\m{1}-\m{A}\right)^{-1}\cdot\vec{y}
 \equiv \m{B}\cdot\vec{y}.
}
Hierbei wird die neu definierte Matrix
$\m{B}=\left(\m{1}-\m{A}\right)^{-1}$ als die
\bfdef{Koeffizientenmatrix des vollen Aufwandes} bezeichnet. Die
Bedeutung ihrer Elemente ist wie folgt:

\maintext{$B_{ij}$ ist die Menge an Wirtschaftsgut $i$, welche zur
Herstellung einer Einheit des Wirtschaftsgutes $j$ vom Sektor $j$ selbst
\textit{und von allen Zuliefersektoren $k$} des Sektors $j$  ben\"otigt
wird. Also die Menge an Wirtschaftsgut $i$, welche zur Bereitstellung 
einer Einheit von $j$ \textit{an den Verbraucher}  ben\"otigt wird. }

%#######################################
\subsection{Veranschaulichung der Koeffizientenmatrix des vollen Aufwandes}
%#######################################

Ist eine Funktion durch eine Reihenentwicklung definiert, wie
$e^x=1+x+\frac{x^2}{2}+\ldots$ oder auch z.B. $\cos(x)$, so sind als Argumente
dieser Funktion nicht nur Zahlenwerte m\"oglich, sondern alle Objekte,
welche sich addieren, multiplizieren und invertieren
lassen, wobei das Ergebnis dieser Operationen in der Objektmenge
bleiben muss. Dies ist insbesondere f\"ur die Menge der
quadratische Matrizen erf\"ullt.\footnote{Genauer muss die sog. ``Gruppeneigenschaft''
bez\"uglich Addition und Multiplikation erf\"ullt sein, das Ergebnis
dieser Operationen muss also wieder ein Element aus der betrachteten Objektmenge sein; dies ist
z.B. bei Vektoren mit dem Skalarprodukt als Multiplikation nicht erf\"ullt.}
Insbesondere gilt dies f\"ur die ``geometrische Reihe''
\bdm
\frac{1}{1-x}=1+x+x^2+x^3+\ldots = \sum\limits_{j=0}^{\infty}x^j.
\edm
Schreibt man die Inverse einer Matrix $\m{M}$ als
$\m{M}^{-1}=''1/\m{M}''$ und $\m{M}\cdot\m{M}=\m{M}^2$,
$\m{M}\cdot\m{M}\cdot\m{M}=\m{M}^3$ usw.,  sowie $\m{M}^0=\m{1}$,
sieht man, dass die L\"osung des
Input-Output-Modells nichts anderes ist als die geometrische Reihe,
angewandt auf die Matrix $A$:
\be
\label{IOMgeomReihe}
\m{B}=\left(\m{1}-\m{A}\right)^{-1}
= ``\frac{1}{\m{1}-\m{A}}''
= \m{1}+\m{A}+\m{A}^2+\m{A}^3+\ldots
= \sum\limits_{j=0}^{\infty}\m{A}^j.
\ee
Im Folgenden wird anhand von zwei Sektoren gezeigt, wie diese
unendliche Reihe die unendliche Sequenz der Kopplungen
repr\"asentiert.

\subsubsection{Beispiel: Zwei-Sektoren-Modell}
Es werden die beiden Sektoren
\bi
\item Sektor 1: Transport und Verkehr,
\item Sektor 2: Maschinenbau
\ei
betrachtet. Ferner werden alle anderen Sektoren zu den externen Verbrauchern,
also zu den Konsumenten $y_i$ von Verkehr und Maschinenbau, gez\"ahlt.
Es soll gelten


\bdm
\m{A}=\frac{1}{4}\myMatrixTwo{1&1\\1&1}.
\edm
Die Sektoren verbrauchen also je 25\% ihrer Produkte f\"ur den
Eigenbedarf und ben\"otigen au\3erdem Produkte des jeweils anderen
Sektors in einer Menge, welche ein Viertel des Wertes der gesamten eigenen Produktion
entspricht.
Fortgesetzte Matrixmultiplikation ergibt f\"ur $n\ge 1$
\bdm
\m{A}^n=\frac{1}{2^{n+1}}\myMatrixTwo{1&1\\1&1}
\edm

und damit z.B.
\bdm
B_{11}
=\left(\sum\limits_{j=0}^{\infty}\m{A}^j\right)_{11}
=1+\frac{1}{4}\sum\limits_{j=0}^{\infty}\left(\frac{1}{2}\right)^j
 = 1+\frac{1}{4\left(1-\frac{1}{2}\right)}=\frac{3}{2},
\edm
und
\bdm
B_{12}=\frac{1}{4}\sum\limits_{j=0}^{\infty}\left(\frac{1}{2}\right)^j = \frac{1}{2}.
\edm
Damit kann man die Kopplungen der Wirtschaftssektoren Verkehr und
Maschinenbau veranschaulichen: Der Koeffizient $B_{11}$ hat die
Anteile
\bdm
B_{11}=\left(\m{1}+\m{A}+\m{A}^2+\m{A}^3+\ldots\right)_{11}
  = 1+A_{11}+\sum\limits_{k=1}^2 A_{1k}A_{k1}
+\sum\limits_{k=1}^2\sum\limits_{l=1}^2A_{1k}A_{kl}A_{l1} + \ldots
\edm
mit den Bedeutungen
\bi
\item Beitrag 1: Eine Einheit an Verkehrsleistung an die Endverbraucher.
\item Beitrag $A_{11}=\frac{1}{4}$: F\"ur eine Einheit an den Verbraucher ben\"otigen die
Verkehrsbetriebe 1/4 an Verkehrsleistungen f\"ur den Eigenbedarf
(Arbeitswege des Personals, 
\"Uberf\"uhrung von Fahrzeugen, Wartungsfahrten, Leerfahrten zur
ersten Einsatzstelle etc).
\item Beitrag $A_{11}^2$: Der
Eigenbedarf erh\"oht die Verkehrsnachfrage, welche wiederum
\textit{zu\-s\"atz\-li\-chen} Eigenbedarf hervorruft. Ein Stra\3enbahnf\"uhrer
f\"ahrt z.B. mit \"OV zum Startpunkt seiner
Schicht; aber auch die Fahrer dieser Verkehrsmittel nehmen
Verkehrsleistung zum Antritt und am Ende \textit{ihrer} Schicht in
Anspruch. 
\item Beitrag $A_{12}A_{21}$: Die Verkehrsmittel altern und
m\"ussen kontinuierlich durch neue ersetzt werden (Anteilige
Auftr\"age  $A_{21}$ an den Maschinenbau). Der Maschinenbau ben\"otigt
f\"ur seinen Betrieb (Mitarbeiterwege, Transport von Teilen etc)
f\"ur eine Einheit  eine Verkehrsleistung von $A_{12}$ Einheiten, also
f\"ur den an die Verkehrsbetriebe gelieferten Bruchteil  $A_{21}$
anteilig die Verkehrsleistung $A_{12}A_{21}$.
\item Die Str\"ome h\"oherer Ordnung werden schnell
un\"ubersehbar. Beispielsweise bedeutet der Beitrag
$A_{12}A_{22}A_{21}$ 
die Verkehrsleistung $A_{12}$ an den Maschinenbau, damit dieser seinen
eigenen Maschinenpark up-to-date h\"alt ($A_{22}$), um schlie\3lich
neue Fahrzeuge an den Verkehrssektor liefern zu k\"onnen ($A_{21}$).
\ei

\aufgabenbox{Aufgabe}{Verdeutlichen Sie sich die Anteile von $B_{12}$.
}

%#######################################
\subsection{Explizite L\"osungen f\"ur zwei und drei Sektoren}
%#######################################

Gl\"ucklicherweise nimmt einen die Inversion einer einzigen Matrix das
Verfolgen der einzelnen Verflechtungen zwischen den Sektoren nach der
Art des letzten Unterabschnitts ab. ``Per Hand'' kann man die Inverse
einer $n\times n$ Matrix 
bis maximal f\"ur $n=5$ sinnvoll mit dem \bfdef{Gau\3'schen
Eliminationsverfahren} l\"osen, numerisch ist die L\"osung auch f\"ur
Hunderte von Sektoren problemlos (der Aufwand steigt mit $n^3$). F\"ur
 $2\times 2$ und $3\times 3$ - Matrizen gibt es einfache explizite
Formeln, vgl. Abschnitt~\ref{sec:matrix}.

\subsubsection{Zwei Sektoren}

Sei $\m{M}$ eine allgemeine invertierbare 
$2\times 2$ Matrix mit den Elementen $M_{ij}$. Dann
gilt nach~\refkl{matrixinv2}
\bdm
\m{M}^{-1}=\frac{1}{\text{det}(\m{M})}
\myMatrixTwo{M_{22} & -M_{12} \\ -M_{21} & M_{11}}
\edm
und damit
\maineq{BtwoTimesTwo}{
\m{B}=\left(\m{1}-\m{A}\right)^{-1}
= \frac{1}{\text{det}(\m{1}-\m{A})}
\myMatrixTwo{1-A_{22} & A_{12} \\ A_{21} & 1-A_{11}}.
}

Das obige Beispiel f\"ur $A_{11}=A_{12}=A_{21}=A_{22}=\frac{1}{4}$
liefert
\bdm
\m{1}-\m{A}=\frac{1}{4}\myMatrixTwo{3&-1\\-1&3}, \quad
\text{det}\left(\m{1}-\m{A}\right)=\frac{1}{16}(9-1)
=\frac{1}{2}, \quad
\m{B}=\frac{1}{2}\myMatrixTwo{3&1\\1&3}.
\edm
also insbesondere die oben mittels der geometrischen Reihe berechneten
Koeffizienten des vollen Aufwands $B_{11}=\frac{3}{2}$ und $B_{12}=\frac{1}{2}$.


\subsubsection{Drei Sektoren}

Hier benutzt man Gleichung~\refkl{matrixinv3}. Alternativ kann man
auch die Eigenschaften von Skalar- und Kreuzprodukten von
Vektoren nutzen, vgl.~\refkl{matrixprodukte}.
Schreibt man die drei Zeilen der Inversen einer invertierbaren
$3\times 3$ Matrix mit Hilfe von drei Zeilenvektoren,
\bdm
\m{M}^{-1}=\myVector{ - \vec{z}_1^T - \\ - \vec{z}_2^T - \\ - \vec{z}_3^T- },
\edm
und die Spalten der Matrix selbst mit Spaltenvektoren,
\bdm
\m{M}=\myMatrixThree{| & | & | \\ \vec{m}_1 & \vec{m}_2 & \vec{m}_3 \\ | & | & | }
\edm
dann gilt
\be
\vec{z}_1 =\frac{\vec{m}_2 \times \vec{m}_3}
{\text{det}\left(\m{M}\right)}, \quad
\vec{z}_2 =\frac{\vec{m}_3 \times \vec{m}_1}
{\text{det}\left(\m{M}\right)}.  \quad
\vec{z}_3 =\frac{\vec{m}_1 \times \vec{m}_2}
{\text{det}\left(\m{M}\right)}.
\ee
sowie
\be
\label{detDrei}
\text{det}\left(\m{M}\right)= \vec{m}_1 \cdot \left(\vec{m}_2 \times \vec{m}_3\right).
\ee
%
Nat\"urlioch muss man zur Bestimmung der Matrix $\m{B}$ 
 f\"ur $\m{M}$ wieder die Differenzmatrix
$\m{1}-\m{A}$ einsetzen.

\aufgabenbox{Aufgabe: Zwei-Sektor-Modell}{Im obigen Beispiel der zwei
Sektoren Verkehr und Maschinenbau mit den direkten Aufwandskoeffizienten
$A_{11}=A_{12}=A_{21}=A_{22}=\frac{1}{4}$ ist bisher die externe Nachfrage
nach Transport-Dienstleistungen doppelt so hoch wie die nach Produkten
des Maschinenbaus. Nun erh\"oht sich die Nachfrage
nach Verkehr um 10\%, w\"ahrend die nach Produkten
des Maschinenbaus unver\"andert bleibt. Um wieviel Prozent m\"ussen
die Gesamtleistungen der beiden Sektoren steigen, um die neue
Nachfrage zu befriedigen?
}

%#######################################
\section{\label{sec:LCA-IOM}Lebenszyklusanalyse und Einbindung des
  Input-Output-Modells}
%#######################################


Ausgehend von einer aktuellen Fragestellung k\"onnte dieser Abschnitt
auch lauten: \textit{Sind E-Fahrzeuge wirklich
  umweltfreundlicher?}. Die Lebenszyklusanalyse wird deshalb anhand
der CO$_2$-Emissionen von Fahrzeugen erl\"autert, sie kann aber f\"ur
beliebige, Ressourcenverbrauch oder Emissionen betreffende
Produkten oder Prozessen eingesetzt werden wie die Emissionen bei Produktion
eines Kilowatts elektrischer Energie aus Kohle- oder Atomkraftwerken,
Windkraft, Wasserkraft, Biomasse, Sonnenzellen oder Solarthermie.

Hintergrund ist, dass
Elektrofahrzeuge im Betrieb \emph{lokal} weder CO$_2$ noch
andere Schadstoffe (au\3er Feinstaub) emittieren, weshalb sie von einigen
Stadtverwaltern sehr geliebt werden, um lokale Umweltprobleme zu
mildern.  Aber geht man hier nicht lediglich nach dem
St-Florians-Prinzip vor? Dies ist insbesondere f\"ur nichtfl\"uchtige
Emissionen, allen voran CO$_2$, relevant. Fr\"uher oder sp\"ater
verteilen sie diese weltweit, egal ob sie im Betrieb lokal
(Verbrennungsmotoren), nichtlokal (Produktion der Kraftstoffe bei Verbrennungsmotoren,
Stromerzeugung bei E-Fahrzeugen, au\3erdem Verschlei\3teile und
Verbrauchsmaterialien), oder in 
vor- oder nachgelagerten Prozessen (Herstellung, Entsorgung)
anfallen. Dies gilt im abgemilderter Sch\"arfe auch f\"ur
konventionelle Kfz mit Verbrennungsmotor, bei der neben den direkten CO$_2$
Emissionen beim Betrieb ebenfalls weitere Emissionen anfallen. 

Man ben\"otigt also eine Analysemethode, welche \emph{alle} Emissionen
erfasst, welche irgendwo oder irgendwann unmittelbar oder mittelbar
durch die Nutzung eines Fahrzeugs (oder allgemein bei der Nutzung
eines ressourcenverbrauchenden und die Umwelt beeinflussenden
Produkts oder Prozesses) anfallen. Dazu werden zwei Methoden
kombiniert:
\bi
\item Die \bfdef{Lebenszyklusanalyse} bzw. 
\bfdef{Life-Cycle-Assessment (LCA)}, auch \bfdef{\"Oko\-bi\-lanz}
genannt, umfasst  welche alle direkt mit der
  Herstellung, dem Betrieb und der Entsorgung anfallenden Emissionen
  umfasst,\footnote{Beispielsweise LCA nach DIN EN ISO 14040. Die
  GEMIS Datenbank enthaelt viele Sachbilanzen und das HBEFA ``Handbuch
  f\"ur Emissionsfaktoren'' die dazugeh\"origen Emissionsfaktoren.}
% 
% Genaueres in BA Marcus Neumann (12.11.13)

\item und die Input-Output-Analse mit dem IOM, welche alle durch
  Verflechtungseffekte anfallenden indirekten Emissionen
  ber\"ucksichtigt.\footnote{Die Abgrenzung ist nicht immer
    scharf. Beispielsweise werden bei der LCA meist nicht nur die
    \emph{Tank-to-Wheel} Emissionen durch die direkte Verbrennung
    ber\"ucksichtigt, sondern auch die \emph{Well-to-Tank} Emissionen durch
    die Produktion und den Transport des Kraftstoffs bis in den
    Tank. Letztere k\"onnte man auch mittels des IOM durch die
    Verflechtung des Sektors ``Treibstoffe'' mit ``Transport'',
    ``Verarbeitung'' und ``Bergbau'' modellieren.} Die LCA wird in das
  Input-Output-Modell meist mittels der hybriden Methodik 
\bfdef{Econometric Input-Output LCA (EIO-LCA)} eingebunden.
\ei
Betrachtet man nur Emissionen pro End-Nachfrage (in Geldeinheiten)
nach Produkten oder Dienstleistungen eines gewissen Sektors, kann man
die EIO-LCA vereinfachen zur IOM-LCA.

%#############################################
\subsection{Lebenszyklusanalyse (LCA)}
%#############################################

Die Lebenszyklusanalyse verl\"auft in mehreren Schritten
(vgl.Abbildung~\ref{fig:LCA}):
%#######################################
\begin{figure}
%\fig{0.9\textwidth}{figsIOM/LCA.eps}
\caption{\label{fig:LCA}Vereinfachtes Flussdiagramm der
  Lebenszklusanalyse 
 eines Kraftfahrzeugs bez\"uglich Emissionen (noch nicht vorhanden,
 vgl. Vorlesungsmitschrift)
} 
\end{figure}
%#######################################

\paragraph{1. Definition der Lebensphasen:}
Bei einem Fahrzeug sind das im Wesentlichen
\bi
\item die Herstellung,
\item der Betrieb
\item und die Entsorgung.
\ei

\paragraph{2. Aufstellung der Sachbilanz:} 
In jeder der Lebensphasen verbraucht einer betrachtete Einheit (hier:
ein Auto) gewisse Mengen verschiedener Ressourcen wie
\bi
\item $y_1\sup{s}$: Masse (kg) an Eisen/Stahl,  
\item $y_2\sup{s}$  Menge (Liter) an Treibstoff (z.B. Benzin),
\item $y_3\sup{s}$: Masse (kg)  an Aluminium,
\item $y_4\sup{s}$: Masse (kg)  an Kunststoffen,
\item $y_5\sup{s}$: Masse (kg)  an Gummi
\ei
und weitere Materialien wie Gummi, Lack, Textilien usw.
\"Uber alle Lebensphasen summiert ergibt dies den
Vektor $\vec{y}\sup{s}$ der \bfdef{Sachbilanz}.

\paragraph{3. Aufstellung der Emissionsfaktoren:} 
Die Bereitstellung der  ben\"otigten Materialien und Ressourcen
ist mit  Emissionen verschiedenster Schadstoffe verbunden.
Dies definiert die sachbilanzbezogene Emissionsfaktoren-Matrix $\m{C}$:

\maintext{Bei $n$ betrachteten
  Emissionsarten (z.B. 1=CO$_2$, 2=Feinstaub, 3=NO$_x$) und $m$ Posten der Sachbilanz (z.B. 1=Stahl, 2=Benzin) ist die
  sachbilanzbezogene Emissionsfaktoren-Matrix $\m{C}$ eine
  unsymmetrische reellwertige  
 $n\times m$-Matrix. Die Komponente $C_{ij}$ gibt an, wieviele
  Einheiten des Schadstoffes $i$ bei der Herstellung einer Einheit des
  Postens $j$ der Sachbilanz \emph{einschlie\3lich der kompletten
    Kette der Vorprodukte} (vgl. Abb.~\ref{fig:WtT}) im Mittel
  anfallen.}

Die erste Zeile der Matrix betrifft also den ersten Schadstoff und
die erste Spalte den ersten Postens der Sachbilanz.
Bei der Bereitstellung eines Liters Benzin ($m=2$) von der F\"orderung bis zur
Tankstelle (\textit{Well-to-Tank}) fallen beispielsweise etwa
\unit[0.3]{kg} an CO$_2$ an. Das entsprechende Matrixelement ist also
(ohne die eigentliche Verbrennung im Motor, siehe Punkt 4)
gleich
\bdm
C_{12}=C_{WtT}=\unit[0.3]{kg CO_2/l}.
\edm
Meist wird hier nicht direkt CO$_2$,
  sondern das sogenannte \bfdef{CO$_2$-\"Aquivalent} angegeben. Dabei
  wird die auf einen kg CO$_2$ bezogene Klimawirksamkeit aller
  Emissionen betrachtet. Auf einen 100-Jahres-Zeitraum
  bezogen, wird beispielsweise Methan 25-fach\footnote{Damit wird auch
    der Grund klar,
    warum  ein Kilogramm Rindfleisch so einen hohen Emissionsfaktor
    bez\"uglich CO$_2$ hat: Vornehm ausgedr\"uckt, sind es die
    Flatulenzen der  K\"uhe.} und Lachgas
  298 fach gewichtet. 

\paragraph{4. Ermittlung der Gesamtemissionen w\"ahrend der Lebenszeit:}

Direkt anhand der Definition der Emissionsfaktorenmatrix ist
offensichtlich, dass die mit der Herstellung der ben\"otigten
Materialien und Verbrauchsstoffe verbundenen Emissionen durch die
Multiplikation der  Emissionsfaktoren-Matrix mit der Materialbilanz 
gegeben sind. Hinzu kommen noch die direkt mit dem Verbrauch (hier von
Benzin) verursachten Emissionen:

\maineq{emissionen-LCA}{ 
\vec{e}\sub{LCA}=\vec{e}\sup{dir} + \m{C}\cdot \vec{y}\sup{s}
}

Bei obiger Definitionen der Sachbilanz sind 
die CO$_2$-Emissionen durch  die erste Komponente gegeben:
\be
\label{emissionenCO2-LCA}
e\sub{CO2}=e_1=e_1\sup{dir} + \sum_{j=1}^n C_{1j} y_j\sup{s}
\ee

Die direkten Emissionen sind bei Verbrennungsmotorfahrzeugen durch
die
\bfdef{Tank-to-Wheel} Emissionen, multipliziert mit der Laufleistung
$L$ w\"ahrend der Lebenszeit, gegeben:
\be
e_1\sup{dir}=C\sub{TtW} L
\ee
Die  Tank-to-Wheel Emissionen ergeben sich dadurch, dass Benzin oder
Diesel im Wesentlichen vollst\"andig verbrennen, dass hei\3t, jedes
Kohlenstoffatom im Treibstoff verbindet sich mit dem Luftsauerstoff zu
einem CO$_2$-Molek\"ul. Damit ergibt sich
\be
C\sub{TtW} = \rho f_c \frac{M\sub{CO2}}{M_C}
\ee
mit der Dichte $\rho$ in kg pro Litern, dem Gewichtsanteil $f_C$ an
Kohlenstoff im Treibstoff und den Molmassen $M\sub{CO2}=\unit[44]{g}$ sowie
$M_C=\unit[12]{g}$. Dies ergibt f\"ur Benzin etwa \unit[2.4]{kg/l} und f\"ur
Diesel etwa \unit[2.65]{kg/l}.

\paragraph{5. Bezug auf eine Leistungseinheit:}
Schlie\3lich werden die Lebenszeit-Emissionen auf eine
Leistungseinheit, hier also auf einen Kilometer Laufleistung, bezogen:
\be
\label{eLCAspez}
\vec{e}\sub{spez}=\frac{1}{L}\vec{e}
\ee
Mit \"ublichen Annahmen ergibt sich bei Kompaktklasse-Benzinfahrzeugen
etwa $e_{1,spez}=\unit[180]{g/km}$, w\"ahrend die direkten Emissionen
bei etwa \unit[140]{g/km} (entspricht einem Verbrauch von 5.8~Liter
Benzin auf 100~km) liegen.

\verstaendnisbox{Wie w\"urde man bei der LCA-Ananlyse eines
  Elektrofahrzeugs vorgehen? Was sind hier die wichtigsten Posten der
  Sachbilanz? Wie hoch sind hier die direkten CO$_2$-Emissionen
  $e_1\sup{dir}$?
}

%#######################################
\begin{figure}
%\fig{0.9\textwidth}{figsIOM/WtT.eps}
\caption{\label{fig:WtT}Die gesamte Kette der Well-to-Tank Emissionen
  bei der Bereitstellung eines Liters an Benzin (noch nicht vorhanden,
 vgl. Vorlesungsmitschrift).
} 
\end{figure}
%#######################################



%#############################################
\subsection{Econometric Input-Output LCA (EIO-LCA)}
%#############################################
\EinsteinBeg

Im Allgemeinen wird das Input-Output-Modell f\"ur volkswirtscahfliche
Fragestellungen angewandt und dabei die gesamte Volkswirtschaft in
Sektoren gegliedert. Gelingte es jedoch, die Sachbilanz auf diese
Sektoren abzubilden, ist auch eine \bfdef{produktbezogene
  Input-Output-Analyse} m\"oglich. Das EIO-LCA verbindet diese
Abbildungen mit einer Lebenszyklusanalyse und umfasst folgende
Schritte:

\paragraph{1. Definition der Lebensphasen:}  Wie bei der LCA.
\paragraph{2. Aufstellung der  Sachbilanz:} 
Im Wesentliche wie bei der LCA. Allerdings sollte man
zus\"atzliche rein dienstleistungsorientierte Aktivit\"aten w\"ahrend
eines Autolebens zus\"atzlich ber\"ucksichtigen, beispielsweise
Reparatur-Dienstleistungen, was hier als \bfdef{erweiterte Sachbilanz}
bezeichnet wird.

\paragraph{3. Abbildung der Sachbilanz auf den IOM-Nachfragevektor:}
Aufgrund der Datenlage kann man die volkswirtschaftlichen Sektoren
nicht frei w\"ahlen, sondern zieht beispielsweise die Einteilung des
Statistischen Bundesamtes in 71
Sektoren heran.

Knackpunkt ist es, die $m$ Posten der Sachbilanz $\vec{y}\sup{s}$ in die $M$ Eintr\"age
des sektorbezogenen Nachfragevektors $\vec{y}$ (bzw, in eine
Untermenge davon) abzubilden. Im
Allgemeinen bezieht sich  ein Posten der Sachbilanz auf mehrere
Sektoren und mehrere Posten der Sachbilanz k\"onnen zur Nachfrage
ein- und desselben Sektors beitragen. Nimmt man an, dass der
Zusammenhang linear-statisch ist, entspricht die Abbildung daher einer
reellwertigen $M\times m$-Matrix $\m{T}$:

\be
\label{y-EIO-LCA}
\vec{y}=\m{T}\cdot \vec{y}\sup{s}
\ee
Neben der eigentlichen Abbildung beinhaltet diese Matrix auch eine
Transformation auf einheitliche Einheiten, also auf Geldeinheiten.

Im obigen Beispiel bildet die erste Spalte von $\m{T}$ den
Sachbilanzposten $y_1^s$: ``Eisen und
Stahl'' auf die volkswirtschaftlichen Sektoren ab. Verwendet man die
73 Sektoren des Bundes, gibt es im Wesentlichen nur einen von null
signifikant unterschiedlichen Eintrag: den Sektor 
 27.10: ``Roheisen, Stahl und Ferrol''. Diese Spalte hat also nur 
einen von null signifikant unterschiedlichen Eintrag,
z.B. \unit[3]{\euro{}/kg}, wenn dies der Preis von einem kg Stahl
ist.  Die zweite Spalte bildet den Sachbilanzposten $y_2^s$:
``Benzin'' auf den Sektor 11:00: ``Erd\"ol, Erdgas und
-produkte'' ab. Der entsprechende Wert des $\m{T}$-Elements ist der
aktuelle Treibstoffpreis in
Euro pro Liter. Schlie\3lich werden die Sachbilanzposten  Aluminium,
Kupfer und Lithium (wichtig als Material f\"ur die Batterie von
Elektroautos) alle auf den Sektor 27.00: ``NE-Metalle'' abgebildet. 
Hier sieht man einen Nachteil des Verfahrens: Materialien mit
unterschiedlichen Emissionen bei der Herstellung (die Herstellung
eines kg an Kupfer impliziert weniger CO$_2$ Emissionen als Aluminium
oder gar Lithium) werden im Wesentlichen auf den selben Sektor
abgebildet und damit gleichbehandelt. \footnote{Die Skalierung auf den
  Preis anstelle auf die Masse lindert diesen Nachteil allerdings
  etwas: Ist Lithium beispielsweise pro kg 10 mal so teuer wie Kupfer,
  w\"urden auch die Emissionen pro kg um den Faktor 10 h\"oher
  berechnet.}

\paragraph{4. Berechnung des Produktionsvektors mit der Standard-IOM:}

Mit der Abbildung \refkl{y-EIO-LCA} der erweiterten Sachbilanz auf den
Nachfragevektor ist der \"Ubergang zur IOM vollzogen und die Abbildung
auf den Produktionsvektor entspricht der Standardabbildung der IOM mit
der Matrix des vollen Aufwandes:
\bdm
\vec{x}=\m{B}\vec{y}=(\m{1}-\m{A})^{-1} \vec{y}
\edm
Bei Fahrzeugen koppelt die Matrix $\m{B}$ den Nachfragevektor insbesondere stark an
den Sektor 38.00: ``elektrische Energie'', auch wenn dieser bei
Benzin-Fahrzeugen nicht direkt in der erweiterten Sachbilanz vorkommt.

\paragraph{5. Ermittlung der Gesamtemissionen:}

Der Emissionsvektor berechnet sich aus dem Produktionsvektor mittels
der sektorbezogenen Emissionsfaktorenmatrix:
\maintext{Die  sektorbezogene
  Emissionsfaktormatrix $\m{D}$ ist eine $n\times M$-Matrix. Die
  Komponente $D_{ij}$ gibt an, wieviele
  Einheiten des Schadstoffes $i$ bei der Herstellung eines Gegenwertes
  von einem Euro des Gutes $j$ \emph{direkt} anfallen. Im Gegensatz
  zur Matrix $\m{C}$ werden die Vorketten \emph{nicht}
  ber\"ucksichtigt, da diese ja in viel vollst\"andigerer Form
  einschlie\3lich aller R\"uckkopplungen durch die Verflechtungsmatrix
  $\m{B}$ abgebildet werden.}

Nach Definition ergibt dies

\maineq{emissionen-EIO-LCA}{
\vec{e}\sub{EIO-LCA}
       =\vec{e}\sup{dir} + \m{D}  \vec{x}
       =\vec{e}\sup{dir} + \m{D} \m{B} \vec{y}
       =\vec{e}\sup{dir} + \m{D} (\m{1}-\m{A})^{-1} \vec{y}
}
%
Diese Hauptformel sieht \"ahnlich aus wie Gl.~\refkl{emissionen-LCA}
der konventionellen LCA. Insbesondere werden die direkten Emissionen
$\vec{e}\sup{dir}$ gleich berechnet. Es gibt aber folgende Unterschiede:
\bi
\item
Der Nachfragevektor $\vec{y}$ bezieht sich auf die volkswirtschaftlichen
Sektoren in Euro anstelle auf den Sachbilanzvektor $\vec{y}\sup{s}$ in physikalischen
Einheiten
\item 
Die Emissionsfaktormatrix $\m{D}$ enth\"alt nur die direkten
Beitr\"age, nicht die Vorketten wie die in $\m{C}$ enthaltenen produktbezogenen
Emissionsfaktoren
\item Daf\"ur ist statt der Einheitsmatrix bei der konventionellen LCA
  die Kopplung der Sektoren untereinander durch die Matrix
  $\m{B}=(\m{1}-\m{A})^{-1}$ ber\"ucksichtigt.
\ei

%#############################################
\subsection{Emissionen bei direkter Verwendung des
  IOM (IOM-LCA)}
%#############################################

Betrachtet man keine Einzelprodukte sondern die durch eine Geldeinheit
an Nachfrage in den verschiedenen Sektoren verursachten Emissionen,
ist es meist g\"unstiger, alle direkten Emissionen in die
Emissionsfaktorenmatrix $\m{D}$ zu stecken. Dazu ist es allerdings
notwendig, dass die Ergebnisse von aggregierten Lebenszyklusanalysen
f\"ur die Produkte bzw. Dienstleistungen der 
betrachteten Sektoren in Form einer sektorbezogenen direkten
Emissionsfaktorenmatrix $\m{D}$ vorliegen. Betrachtet man beispielsweise
den Sektor des \"OPNV, werden die direkten Emissionen beim Betrieb der Busse
oder Diesel-Loks (elektrifizierte Verkehrsmittel haben ja keine
direkten Emissionen sondern nur indirekte \"uber die
Verflechtungseffekte bzw. Vorketten!) auf den Fahrpreis
umgelegt. Formel~\refkl{emissionen-EIO-LCA} vereinfacht sich dann zu
\maineq{emissionen-IOM}{
\vec{e}\sub{IOM}
       =\m{D}  \vec{x}
       =\m{D} \m{B} \vec{y}
       =\m{D} (\m{1}-\m{A})^{-1} \vec{y}
}

%
%Mit Hilfe dieser Methoden werden folgende Gr\"o\3en ermittelt:
%\bi
%\item Vom Endverbraucher (=Autofahrer) nachgefragte Mengen $\vec{y}$
%  an Waren und Dienstleistungen (in Euro) f\"ur Erwerb, Betrieb und Entsorgung
%  eines Autos \emph{w\"ahrend des gesamten Autolebens}, z.B. bei
%  Aufgliederungen in die Sektoren 1=Fahrzeuge, 2=Treibstoffe,
%  3=Sonstiges:\footnote{Nat\"urlich muss man bei einem Vergleich 
%    E-Fahrzeuge \emph{vs.} Fahrzeuge mit Verbrennungsmotoren die
%    hier sehr wichtigen Sektoren ``Elektrische Energie'' und
%    ``Batterien'' separat betrachten, Dies wird hier der Einfachheit
%    halber nicht gemacht.}
%\bdm
%\vec{y}\tr=(\text{1 Auto}, \text{\unit[10\,000]{l} Benzin},
%  \text{Sonstiges wie Reparaturen+Versicherungen})
%\edm
%bzw. entsprechende Geldbetr\"age
%\bdm
%\vec{y}\tr=(\unit[20\,000]{\eur}, \unit[15\,000]{\eur}, \unit[15\,000]{\eur})
%\edm
%\item Matrix des direkten Aufwandes $\m{A}$, im Beispiel also eine
%  $3\times 3$ Matrix,
%\item Emissionsfaktormatrix $\m{D}$. Jede Spalte $j$ gibt dabei die
%  spezifischen Emissionen an Schadstoff $j$ an, welche \emph{direkt} durch die
%  Herstellung bzw. der ordnungsgem\"a\3e Nutzung
%  (Betriebsstoffe, Kraftstoffe) einer Einheit der verschiedenen
%  Sektoren anfallen. Gibt die erste Spalte $\vec{c}$ von $\m{D}$
%  beispielsweise die CO$_2$-Emissionen an, so k\"onnte $\vec{c}$ im obigen
%  Beispiel wie folgt aussehen:\footnote{Die zweite Komponente $e_2$ folgt
%    direkt aus der Chemie der Verbrennung: \unit[2.4]{kg
%      CO$_2$/Liter}; bei einer feineren Aufgliederung in Sektoren
%    w\"urde bei der ersten Komponente $e_1=0$ stehen, da die
%    Herstellung des Autos direkt keinen Verbrauch impliziert, wohl
%    aber die dabei verwendeten Materialien, die Montage, die n\"otigen
%    Transporte, Fabrikhallen und B\"urogeb\"aude. Man k\"onnte auch
%    bei drei Sektoren $e_1=0$ setzen und alle Emissionen in den ``Sonstiges''
%    Sektor verlagern.}
%\bdm
%\vec{c}\tr=(\unit[10\,000]{kg/Auto}, \unit[2.4]{kg/l}, 0)
%\edm
%bzw. in kg/Euro bei einem auch oben angenommenen Preis von \unit[1.5]{\eur/l}:
%\bdm
%\vec{c}\tr=(\unit[0.5]{kg/\eur}, \unit[1.6]{kg/\eur}, 0).
%\edm
%\ei
%

\subsection{Konsistenz der Methode bei Preissteigerungen}

Da man, wie bereits in Abschnitt~\ref{sec:IOMdef} erw\"ahnt, alle
G\"uter und Dienstleistungen in kommensurable Einheiten darstellen
muss, werden alle Gr\"o\3en auf den ``kleinsten gemeinsamen Nenner''
einer Geldeinheit (\eur) gebracht. Nun k\"ummert sich die Physik der
Emissionen nicht um etwas so Menschliches und Ver\"anderliches wie
Preise.\footnote{Abgesehen davon, dass der Mensch selbst als Reaktion
  auf Preissteigerungen andere Produktionsmethoden verwendet,
  beispielsweise bei starkem \"Olpreisanstieg eine vermehrte Nutzung sogenannter
  ``nichtkonventioneller'' F\"orderungsmethoden wie \"Ol aus
  Teers\"anden, welche deutlich mehr Energie und CO$_2$ Emissionen pro
  Liter implizieren. Dies w\"urde aber im Rahmen des IOM durch eine \"Anderung der
  $\m{A}$-Matrixelemente ber\"ucksichtigt werden.}
Folglich m\"ussen die nach~\refkl{emissionen-EIO-LCA} berechneten Emissionen
unabh\"angig von den Preisen der einzelnen Waren und Dienstleistungen
sein. Nun sind aber alle Matrizen und Vektoren des IOM und alle auf
den rechten Seiten von~\refkl{emissionen-EIO-LCA}
 vorkommenden Gr\"o\3en auf
Preiseinheiten bezogen, so dass sich diese mit den Preisen
\"andern. Im Folgenden zeigen wir, dass die Emissionen aber
gegen\"uber Preis\"anderungen invariant sind, das
Modell~\refkl{emissionen-EIO-LCA} also in dieser Hinsicht konsistent ist.

Wir beschreiben die Preis\"anderungen in den verschiedenen Sektoren
durch einen Faktorvektor $\vec{f}$ bzw. eine quadratisch-diagonale Faktormatrix $\m{F}$,
\be
\vec{f}=\myVector{\text{Preissteigerungsfaktor Sektor 1}\\
\text{Preissteigerungsfaktor Sektor 2}\\ \vdots}, \quad
\m{F}=\text{diag} \ \vec{f}=
\myMatrixThree{f_1 & 0 & \hdots\\0 & f_2 & \\ \vdots & & \ddots}, 
\ee
welche die Preissteigerungen einer physikalischen
bzw. organisatorischen Einheit der verschiedenen Sektoren beschreiben.
Bei gleichem Verbrauch \"andert sich der Endverbrauchsvektor $\vec{y}$
gem\"a\3
\bdm
\vec{y} \ \to \ \vec{y}^*=\m{F}\vec{y},
\edm
w\"ahrend man direkt aus der Definition folgern kann, dass 
die neue Koeffizientenmatrix des direkten Aufwands aus der alten
hervorgeht, indem man die Zeilen $i$ mit $f_i$ und die Spalten $j$ mit
$1/f_j$ multipliziert:
\bdma
A_{ij} &=& \frac{x_{ij}}{x_j} \ \Rightarrow \\
A^*_{ij} &=& \frac{x^*_{ij}}{x^*_j} \\
   &=& \frac{f_i x_{ij}}{f_j x_j} = \frac{f_i}{f_j} A_{ij}
\edma
Dies kann kompakt durch
\bdm
\m{A} \ \to \ \m{A}^*=\m{F}\m{A}\m{F}^{-1}
\edm
ausgedr\"uckt werden (Beweis durch Nachrechnen). Zur Berechnung der neuen Koeffizientenmatrix des
vollen Aufwands wendet man sinnvollerweise die
Reihenentwicklung~\refkl{IOMgeomReihe} an:
\bdm
\m{B}^*=\left(\m{1}-\m{A}^*\right)^{-1}
= \m{1}+\m{A}^*+(\m{A}^*)^2+(\m{A}^*)^3+\ldots
\edm
Nun gilt
\bdma
(\m{A}^*)^2 &=&\m{F}\m{A}\m{F}^{-1}\m{F}\m{A}\m{F}^{-1}=\m{F}\m{A}^2\m{F}^{-1},\\
(\m{A}^*)^3
&=&\m{F}\m{A}\m{F}^{-1}\m{F}\m{A}^2\m{F}^{-1}=\m{F}\m{A}^3\m{F}^{-1},\\
\vdots  &=& \vdots
\edma
also (unter Ber\"ucksichtigung der Assoziativit\"at)
\bdma
\m{B}^* &=&\sum\limits_{j=0}^{\infty}(\m{A}^*)^j\\
 &=& \sum\limits_{j=0}^{\infty}\m{F}\m{A}^j\m{F}^{-1}\\
 &=& \m{F} \left(\sum\limits_{j=0}^{\infty}\m{A}^j\right)\m{F}^{-1}\\
 &=& \m{F} \m{B} \m{F}^{-1}.
\edma
Schlie\3lich m\"ussen als Folge der Preis\"anderungen alle Spalten $j$ der
Emissionsfaktormatrix mit $1/f_j$ multipliziert werden, da man bei
Preissteigerungen pro
Euro weniger vom Gut $j$ produziert/verbraucht und damit auch
weniger pro Euro emittiert:
\bdm
\m{D}^*=\m{D}\m{F}^{-1}
\edm
Damit ergibt sich nach~\refkl{emissionen-EIO-LCA} der Emissionsvektor nach Preissteigerungen zu
\bdma
\vec{e}^* - \vec{e}\sup{dir} &=& (\m{D}^*)\m{B}^*\vec{y}^*\\
  &=& \m{D}\m{F}^{-1}\m{F}\m{B}\m{F}^{-1}\m{F}\vec{y}\\
   &=& \m{D} \m{B}\vec{y}.
\edma
Die Emissionen sind also, wie gefordert,  invariant bez\"uglich
Preis\"anderungen, obwohl sich alle beteiligten Gr\"o\3en \"andern. 



%#######################################
\section{Dynamische Verflechtungsmodelle}
%#######################################
\EinsteinBeg

Das klassische Verflechtungsmodell setzt instantane Anpassungen der
Leistungen der einzelnen Sektoren an eine ver\"anderte Nachfrage
voraus, so dass zu jedem Zeitpunkt das System bez\"uglich der
variablen Nachfrage im Gleichgewicht ist, also durch \refkl{IOM}
beschrieben wird. Dies ist im Allgemeinen unrealistisch:
\bi
\item F\"ur h\"ohere Produktion muss man neue Mitarbeiter einstellen,
bei geringer Nachfrage entlassen,
\item Maschinen und andere Investitionsg\"uter k\"onnen ebenfalls
nicht ``\"uber Nacht'' besorgt oder verkauft/abgeschrieben
werden. W\"ahrend einer \"Ubergangszeit muss deshalb eine gestiegene
Nachfrage durch Abbau von Lagerbest\"anden befriedigt werden.
\ei

Bestenfalls k\"onnen die Verantwortlichen \"uber die
\textit{Steigerungsrate} $\abl{x_i}{t}$ des Produktionsaussto\3es disponieren.
Ist das System nachfrageorientiert, spielt also eine eventuelle
Knappheit an Zulieferprodukten oder Rohstoffen keine Rolle, 
wird die Produktion hochgefahren (die
Steigerungsrate ist positiv), wenn die Diskrepanz zwischen der
externen Nachfrage $y_i$ und dem beim momentanen Systemzustand
f\"ur diese Nachfrage im Flie\3gleichgewicht zur Verf\"ugung stehenden  Angebot
$y_i\sup{eff}=x_i-\sum_jA_{ij}x_j$ positiv ist: ``Die Nachfrage \"ubersteigt das
Angebot''. Andernfalls wird die Produktion heruntergefahren.

Im einfachsten Fall wird dies linear modelliert und f\"ur jeden Sektor
eine charakteristische Anpassungszeit $\tau_i$ angenommen, welche in
der Gr\"o\3enordnung von Monaten oder Jahren liegt und von der
``Schnelllebigkeit'' der Investitionsg\"uter des jeweiligen Sektors abh\"angt. Dies f\"uhrt zu
folgendem linearen, dynamischen und gekoppelten 
Modell\footnote{Dies
ist nicht das klassische dynamische Leontief-Modell (LIOM), welches nur bei
exponentiellem Wachstum funktioniert und ansonsten unrealistische
instabile L\"osungen liefert. F\"ur das LIOM, siehe z.B. Holub/Schnabl,
``Input-Output-Rechnung'', Oldenbourg (1994), S. 556 ff, insbesondere
S. 592.}
%#####################
\maineq{IOMdyn}{
\abl{x_i}{t}=\frac{1}{\tau_i}
 \left[y_i-\left(x_i-\sum\limits_{j=1}^nA_{ij}x_j\right)\right]
 = \frac{1}{\tau}\left(y_i-y_i\sup{eq}\right)
}
%######################
In der Matrix-Vektor-Notation wird dies mit der Anpassungszeitmatrix
\be
\vec{\tau}=\myMatrixFour{\tau_1 & 0 & \cdots & \\
 0 & \tau_2 &  & \vdots \\
  \vdots & & \ddots & 0\\
  & \cdots & 0 & \tau_n}
\ee
zu der kompakten Modelldarstellung
%#####################
\maineq{IOMdynMat}{
\vec{\tau}\cdot \abl{\vec{x}}{t}=\vec{y}-\left(\m{1}-\m{A}\right)\cdot\vec{x}.
}
%#####################


\subsection*{Eigenschaften}
\bi
\item Bei konstanter Nachfrage, $\abl{y_i}{t}=0$, n\"ahert sich das
System dem durch das
klassische statische Modell \refkl{IOM} gekennzeichnete Gleichgewicht
nach einer Zeit an, welche in etwa der l\"angsten
Einzelzeitkonstante, $\max(\tau_i)$, entspricht.
\item Bei einen und zwei Sektoren sind keine Schwingungen m\"oglich,
auch keine ge\-d\"ampf\-ten.
\ei
Mathematisch stellt das Modell ein System von linearen
Differenzialgleichungen dar, welches mit den Methoden der
linearen Algebra unschwer, aber h\"aufig m\"uhsam l\"osbar ist.
Im Rahmen dieser Vorlesung werden allgemeine analytische 
L\"osungen nur f\"ur  der eindimensionalen
Fall vorgestellt und ansonsten einige fertig gerechnete zweisektorale
L\"osungen als Beispiel vorgestellt.


%#######################################
\subsection{Spezialfall: Dynamisches Ein-Sektor-Modell}
%#######################################

\newcommand{\tiltau}{\tilde{\tau}}
\newcommand{\tily}{\tilde{y}}

Im dynamischen Ein-Sektor-Modell gibt es nur einen einzigen Waren und
Dienstleistungen produzierenden Sektor (welcher die
gesamte Wirtschaft darstellt), sowie die Verbraucher. Mit $x_1=x$,
$y_1=y$ und $A_{11}=a$ reduziert sich das Modell
\refkl{IOMdyn}  zu
\bdm
\abl{x(t)}{t}=\frac{1}{\tau}\left[y(t)-(1-a)x(t)\right].
\edm
Setzt man den eindimensionalen Koeffizient des vollen Aufwandes
$b=1/(1-a)$ sowie
$\tiltau=b\tau$ und $\tily=by$, so wird dies zu
\be
\label{IOMdynSingle}
\abl{x(t)}{t}=\frac{1}{\tiltau}\left[\tily(t)-x(t)\right].
\ee
Hat man zur Zeit $t=0$ den Ausgangszustand $x(0)=x_0$, so ist die
L\"osung dieser Differenzialgleichung in Abh\"angigkeit der
Nachfragekurve $\tily(t)=by(t)$ f\"ur $t\ge 0$ gegeben durch
\maineq{IOMdynSingleLoes}{
x(t)=x_0e^{-\frac{t}{\tiltau}}+\frac{1}{\tiltau} \int\limits_0^{t}
e^{-\frac{t-t'}{\tiltau}}\tily(t')\diff{t'}.
}
Insbesondere gilt f\"ur eine zur Zeit $t=0$ sprunghaft vom Wert $y_a$
auf $y_e$ sich \"andernde Nachfrage die L\"osung
\be
x(t)=\left(x_0-by_e\right) e^{-\frac{t}{b\tau}} + by_e.
\ee
Die Produktion passt sich also mit der Zeitkonstante $\tiltau=b\tau
\ge \tau$ dem neuen, durch das klassische Ein-Sektor IOM gegebenen
Gleichgewichtswert $by_e$ an. Die Zeit $\tiltau$ ist etwas gr\"o\3er
als $\tau$, da sich die durch die erh\"ohte Produktion erh\"ohte
interne Nachfrage mittels der Strategie ``Steigerung
proportional zur Differenz zwischen externer Nachfrage und Nettoangebot''
sozusagen ``herumsprechen'' muss.
 Herrscht insbesondere bereits vor dem
Sprung Gleichgewicht, $x_0=by_a$, so ist das Ergebnis durch 
\bdm
x(t)=b\left(y_e+(y_a-y_e)e^{-\frac{t}{b\tau}}\right)
\edm
gegeben (vgl. Abbildung \ref{fig:IOMdyn1}).


%###############################
\begin{figure}
\fig{0.65\textwidth}{figsIOM/IOMdyn1.eps} \vspace{-4mm}

\caption{\label{fig:IOMdyn1}Dynamisches Einsektormodell: Anpassung der
Gesamtproduktion und des der Nachfrage zur Verf\"ugung
Angebotes, wenn die Nachfrage sich zum Zeitpunkt $t=0$ pl\"otzlich um
20\% \"andert. ($A_{11}=a=0.2$, $\tau=1$).
}
\end{figure}
 %###############################

%#######################################
\subsection{Varianten dynamischer Verflechtungsmodelle}
%#######################################

%#######################################
\subsubsection{Begrenzung durch das Angebot}
%#######################################

Das durch Gl. \refkl{IOMdyn} beschriebene dynamische IOM ist
nachfragebestimmt, d.h. die Produktionssteigerungen sind proportional
der Differenz aus Nachfrage $y_i$ und Nettoangebot
$((\m{1}-\m{A})\cdot \vec{x})_i$. Bisweilen ist jedoch die Produktion
nicht durch die Nachfrage begrenzt,  sondern durch den Mangel an
Rohstoffen oder anderen ben\"otigten Zulieferprodukten.

\paragraph{Beispiel f\"ur zwei Sektoren}

Zum Herstellen von Solarzellen ben\"otigt man Siliziumkristalle
einer bestimmten Spezifikation. Obwohl es Silizium (ziemlich wortgetreu) ``wie
Sand am Meer'' gibt, ist die Verarbeitungsstufe zum Zustand, wie er
bei Solarzellen ben\"otigt wird, nur mit aufw\"andigen
Investitionsg\"utern durchzuf\"uhren. Bezeichnet man den
Rohstoff verarbeitenden Sektor (Silizium) als Sektor 1 und den Sektor
alternativer Energieerzeugung (Solarzellen) als Sektor 2,
hat Sektor 1 also eine hohe
Zeitkonstante $\tau_1$. Folgende, das Prinzip des Mechanismus 
nicht beeinflussende weitere
Annahmen werden getroffen: Weder der Endverbraucher
noch der Siliziumsektor  ben\"otigen selbst Silizium,  $y_1=A_{11}=0$.
Solarzellen werden nur
vom Endverbraucher nachgefragt  (die Sektoren benutzen f\"ur ihre
Energiebedarf konventionelle Kraftwerke), also $A_{21}=A_{22}=0$.
Damit gilt im nachfragebegrenzten Zustand das dynamische
IOM,
\be
\label{IOMdynDemand}
\begin{array}{rcl}
\tau_1 \abl{x_1}{t} &=& -x_1 + A_{12} x_2, \\
\tau_2 \abl{x_2}{t} &=& -x_2 + y_2.
\end{array}
\ee
In der Gleichung f\"ur $\abl{x_2}{t}$ kommt $x_1$ gar nicht vor, da
der Sektor $x_1$ die Produkte von Sektor 2 nicht ben\"otigt, also
f\"ur die Nachfrage irrelevant ist. Dass Sektor 2 wiederum das
\textit{Angebot} von Sektor 1 zu seiner Produktion ben\"otigt, wird
nicht ber\"ucksichtigt, woran man die Nachfrageorientierung dieses
Ansatzes sieht.

Im angebotsbegrenzten Zustand, wie 2006 und 2007 bei den Solarzellen, ist die Produktion
$x_2$ jedoch von der Verf\"ugbarkeit geeigneten
Siliziums begrenzt, so dass nur \"uber kurze Zeit (Verwendung von
Lagerbest\"anden) der Verbrauch $A_{12}x_2$ an Silizium den Zufluss $x_{12}=x_1$ (alles
Silizium geht hier  an die Solarzellenhersteller, da es f\"ur Sektor 1
keinen Endabnehmer gibt) \"uberschreiten kann.
Damit kann der Solarzellenaussto\3 
$x_2$ nur kurzfristig (Zeitskala $\tau_2$) den Wert
$x_1/A_{12}$ \"ubersteigen und Gleichung \refkl{IOMdynDemand} 
f\"ur $x_2$ wird zu
\be
\label{IOMdynSupplyTwo}
\tau_2 \abl{x_2}{t} =-x_2+\frac{1}{A_{12}} x_1.
\ee
Die Nachfrage $y_2$ der Endabnehmer geht nun (\"uber das Dr\"angen
von Sektor 2 an Sektor 1, doch mehr Silizium zu erzeugen) indirekt auf
Sektor 1 \"uber, so dass f\"ur diesen Sektor, der nachfragegetrieben
bleibt, nun bei der Ber\"ucksichtigung der Nachfrage von Sektor 2 nicht dessen
 aktuellen Produktion $x_2$, sondern die nach dem
statischen IOM \textit{gew\"unschte} Produktion $x_2=y_2$ relevant
ist. Damit wird die Gleichung \refkl{IOMdynDemand} 
f\"ur $x_1$ zu
\be
\label{IOMdynSupplyOne}
\tau_1 \abl{x_1}{t} = -x_1 + A_{12} y_2.
\ee
Die kombinierte Dynamik kann jederzeit zwischen den
f\"ur $y_2<=x_1/A_{12}$ g\"ultigen nachfragebegrenzten Zustand \refkl{IOMdynDemand}
und den sonst g\"ultigen und durch   \refkl{IOMdynSupplyTwo} und
\refkl{IOMdynSupplyOne} definierten
angebotsbegrenzten Zustand wechseln.

%###########################
\begin{figure}
\fig{0.7\textwidth}{figsIOM/IOMdyn2.eps}
\vspace{-7mm}

\caption{\label{fig:IOMdynSupply} Reaktion der Sektoren 1 (Silizium)
und 2 (Solarzellen) auf eine pl\"otzliche Nachfragesteigerung um 100 \%.
}
\end{figure}
%###########################

\paragraph{Zahlenbeispiel}

Es sei $\tau_1=2$ und $\tau_2=0.5$ (Zeiteinheiten, z.B. Jahre). Die
H\"alfte des Wertes der Solarzellen 
bestehe in den eingekauften Siliziumkristallen, also  $A_{12}=0.5$. Im
Ausgangszustand sei das System bei einer Nachfrage $y_2=1$ im durch
das statische IOM gegebenen Gleichgewicht, also $x_2=2x_1=y_2=1$. Zum
Zeitpunkt $t=0$ steigt die Nachfrage sprunghaft von 1 auf 2.
Da $\tau_1>\tau_2$ will Sektor 2 die Produktion schneller als Sektor 1
hochfahren, so dass er in den angebotsbegrenzten Zustand
ger\"at. Die L\"osungen lauten (vgl. Abb. \ref{fig:IOMdynSupply})
\bdma
x_1 &=& 1-\frac{1}{2} e^{-\frac{t}{2}}, \\
x_2 &=& 2 - \frac{1}{3}e^{-2t} - \frac{2}{3} e^{-\frac{t}{2}}.
\edma
%
Zum Vergleich lautet die L\"osung f\"ur $x_2$ beim rein
nachfragegetriebenen dynamischen IOM
\bdm
x_2 = 1-\frac{1}{2} e^{-2t}.
\edm


\subsubsection{Verallgemeinerung auf mehr als zwei Sektoren}

Bei mehreren Sektoren hat der Hersteller des knappen Gutes (wieder als
Sektor 1 bezeichnet) im
Allgemeinen die Wahl, mit welchem maximalen Anteil $P_{1j}$ seiner Produktion
er die Sektoren $j$ und mit welchem maximalen Anteil $P_{1y}$ er die
externe Nachfrage bedient. (nat\"urlich gilt $\sum_jP_{1j}+P_{1y}=1$).
Damit k\"onnen die anderen Sektoren $j$ ihre 
Brutto-Nachfrage $\sum_k A_{jk} + y_j$ nur dann bedienen, wenn diese
die Zulieferbegrenzung  $P_{1j}/A_{1j}x_1$ nicht \"ubersteigt. 

Generell kann sich jeder Sektor getrennt von den anderen im Nachfrage-
oder Angebotsbestimmten Zustand befinden und sich dies auch jederzeit
\"andern, was das allgemeine System schnell un\"ubersichtlich
 macht.\footnote{Vielleicht ist es deshalb so schwierig, zu bestimmen,
ob der Ansatz von Keynes (die Nachfrage
ist marktbestimmend) zutrifft oder nicht.}

%#######################################
\subsection{Produktionsketten (Supply Chains)}
%#######################################


Oft hat man Produktionsketten, bei denen der Endverbraucher erst
\"uber eine Reihe von Zwischenstufen erreicht wird. Da bei den
Zwischenstufen (Vertrieb) nichts produziert sondern nur
zwischengelagert wird sowie die Lagerkapazit\"at endlich ist, 
ist es sinnvoll, die $x_{ij}$ als
Waren\emph{str\"ome} $x_{ij}$ aufzufassen und dar\"uber hinaus noch
die Menge $W_i$ des in jeder Stufe $i$ vorr\"atigen 
\textit{Bestands} an Waren bzw. G\"utern
 in Lagern, Verkaufsr\"aumen etc. als weitere
dynamische Variable einzuf\"uhren. Die folgenden \"Uberlegungen werden
einfacher, wenn wir $W_i$ nicht als die Warenmenge, sondern als
\emph{Differenz} zum angezielten Lagerbestand auffassen. Positive
Werte bezeichnen also zu hohe und negative Werte zu geringe Lagerbest\"ande.

%###############################
\begin{figure}
\fig{0.8\textwidth}{figsIOM/supplyChainSketch.eps} \vspace{-4mm}

\caption{\label{fig:supplyChainSketch}Materialfluss vom Produzenten
\"uber den Vertrieb zum Verbraucher. 
}
\end{figure}
 %###############################

Wir betrachten nun der \"Ubersichtlichheit halber die konkreten lineare 
Warenflussbeziehungen der Abb.~\ref{fig:supplyChainSketch}. 
Zun\"achst gelten folgende unmittelbar
anschauliche Kontinuit\"ats\-be\-ziehung zwischen den Waren $W_i$ in den
Lagern des Sektor $i$ und den Warenstr\"omen:
\be
\label{supplyChain-kont}
\abl{W_2}{t}=x_{12}(t)-x_{23}(t), \quad
\abl{W_3}{t}=x_{23}(t)-y(t).
\ee
Des Weiteren versuchen die Warendisponenten der jeweiligen Sektoren,
zu niedrige oder zu hohe Lagerbest\"ande durch Hoch-
bzw. Herunterfahren der Warenbestellungen vom jeweils vorgelagerten
Sektor zu kontrollieren, w\"ahrend die nachgelagerten Sektoren als
nicht weiter beeinflussbare Nachfrager auftreten. Die \"Anderung des
Warenflusses vom vorgelagerten Sektor kann jedoch nicht augenblicklich
geschehen, da beim Hochfahren erst Ressourcen f\"ur die erh\"ohten
Warenfl\"usse (bzw. beim ersten Sektor f\"ur die Produktionsrate $x_1$)
bereitsgestellt werden m\"ussen und au\3erdem zwischen
Bestellung und Lieferung eine endliche Zeitspanne liegt. 

Damit kann man die Warenstr\"me nur allm\"ahlich hoch- oder
herunterfahren. Es wird nun folgenderma\3en disponiert:
\bi
\item Ist zu wenig in den Lagern ($W_i<0$), wird der eingehende
Warenstrom hochgefahren, und zwar umso mehr, je h\"oher der
Fehlbestand ist. 
\item Analog werden bei zu hohem Bestand die Bestellungen
heruntergefahren.
\ei
In der Sprache der Mathematik lautet dies:
\be
\label{supplyChain-disponieren}
\abl{x_{12}}{t}=-\beta_2 W_2, \quad
\abl{x_{23}}{t}=-\beta_3 W_3.
\ee
Hierbei geben die Parameter $\beta_i$ an, wie schnell
die Sektoren auf Bestellungen reagieren k\"onnen. Leitet man nun die
Gleichungen von~\refkl{supplyChain-disponieren} noch einmal nach der
Zeit ab, so kann man mit~\refkl{supplyChain-kont} die Variablen der
Warenbest\"ande eliminieren und erh\"ahlt ein dynamisches
Verflechtungsmodell, was f\"ur diesen speziellen Fall
\bfdef{Supply-Chain-Modell} genannt wird:
\be
\label{supplyChain-dyn}
\ablzwei{x_{12}}{t}=-\beta_2 (x_{12}-x_{23}), \quad
\ablzwei{x_{23}}{t}=-\beta_3 (x_{23} - y).
\ee
Nimmt man nun an, dass die Nachfrage $y(t)$ mit der Amplitude $A_y$ und
der Periode $2\pi/\omega$ schwankt, also
\bdm
y(t)=\bar{y}+A_y\cos \omega t,
\edm
und naturgem\"a\3 alle Warenstr\"ome mit derselben Periode schwanken
(das System ist ja nachfragegetrieben!), so zeigt direktes Einsetzen, 
dass sich die Amplitude der Schwankungen
aufschaukelt, wenn man in der Produktionskette
(\emph{Supply Chain}) ``nach vorne'', also Richtung Hersteller
geht. Insbesondere haben die Warenstr\"ome $x_{23}$ und $x_{12}$ die
Amplituden 
\bdm
A_{23}=\left|\frac{A_y}{1-\frac{\omega^2}{\beta_3}}\right|, \quad
A_{12}=\left|\frac{A_{23}}{1-\frac{\omega^2}{\beta_2}}\right|
= \left|\frac{A_y}{\left(1-\frac{\omega^2}{\beta_2}\right)
   \left(1-\frac{\omega^2}{\beta_3}\right)}\right|.
\edm
%##########################
\begin{figure}
\fig{0.7\textwidth}{figsIOM/bullwhip.eps}
\caption{\label{fig:bullwhip}Der Bullwhip-Effekt bei der Distribution von Bier
  an die Bierg\"arten}
\end{figure}
%##########################

Abbildung~\ref{fig:bullwhip} zeigt dies am Beispiel der Supply Chain des
Bieres von der Brauerei \"uber den Gro\3handel zur Kneipe bzw. Biergarten 
 bis zur durstigen
Kehle des ``Endverbrauchers''. Der Aussto\3 der Endabnehmer schwankt
w\"ochentlich (z.B. mit Peaks an den Freitagen oder Samstagen). Die Lieferzeit
$T_3=2\pi/\omega_3$ des Gro\3handels an die Kneipe betr\"agt im Beispiel
\unit[5]{Tage} und die der Brauerei an den Gro\3handel
$T_2=2\pi/\omega_2=\unit[9]{Tage}$. Man sieht, dass bei Lieferzeiten l\"anger
als die Schwankungsperiode der Nachfrage sich die Schwankungen im Vorzeichen
umkehren, ansonsten nicht.

\subsubsection*{Diskussion}
\bi
\item Sind die Nachfrageschwankungen sehr langsam, ist $\omega$ sehr
viel kleiner  als die durch $\sqrt{\beta_i}$ charakterisierten
Reaktionsschnelligkeiten der ``Kettenglieder'' der Supply Chain
($1/\sqrt{\beta_i}$ sind die Reaktionszeiten). Dann kommt
die Produktionskette mit den Schwankungen mit und die Amplituden
vergr\"o\3ern sich kaum.
\item Sind die Nachfrageschwankungen sehr schnell,  $\omega$ also
deutlich gr\"o\3er als alle $\sqrt{\beta_i}$, nehmen die Schwankungen
ab: Ehe die Disponenten ihre Warenstr\"ome deutlich hoch- oder
herunterfahren k\"onnen, hat sich die Nachfrageschwankung schon ins
Gegenteil verkehrt.
\item  Sind die Nachfrageschwankungen hingegen von der
Gr\"o\3enordnung der Reaktionszeiten der Sektoren, $\omega$ also von
\"ahnlicher Gr\"o\3e wie die $\sqrt{\beta_i}$, steigen die
Schwankungen rapide an. Dies bezeichnet man auch als
\bfdef{Bullwhip-Effekt.}
\ei
\aufgabenbox{Aufgabe: Geht es besser?}{Was k\"onnte der Lagerdisponent
unternehmen, um den Bullwhip-Effekt zu vermeiden?}
{\scriptsize Antwort: (1) vorhersagbare Zyklen (im Bier-Beispielwie der Wochenzyklus und auch
 ein Jahreszyklus) durch
  sogenannte \emph{Ganglinien} historischer Daten antizipieren.
(2) Auf unregelm\"a\3igere Schwankungen (beispielsweise durch Sch\"on-
bzw. Schlechtwetter) antizipativer disponieren und beispielsweise die
Bestellungen nicht nur nach dem Lager\emph{bestand}, sondern zus\"atzlich
in Abh\"angigkeit der aktuellen \emph{\"Anderungen} des Bestands
hoch oder herunterfahren. (Dies ergibt in den Gleichungen zus\"atzliche
``D\"ampfungsterme'', welche die Schwankungen m\"a\3igen.)}
%
%%###############################
%\begin{figure}
%\fig{0.65\textwidth}{figsIOM/supplyChain1.eps} \vspace{-4mm}
%
%\caption{\label{fig:supplyChain1}Supply Chain eines Sektors 1, welcher
%nur f\"ur den Vertrieb (Sektor 2) produziert. Dargestellt ist die
%L\"osung der Gleichung \protect\ref{supplyChainTwo} mit
%$\tau_1=\tau_2=1$ f\"ur den Fall,
%dass das System anfangs ($t<0$) bei einer Nachfrage von $y_2=1$ im
%Gleichgewicht ist und die Nachfrage zur Zeit $t=0$ pl\"otzlich auf
%$y_2=2$ steigt.
%}
%\end{figure}
% %###############################
%
%Abbildung \ref{fig:supplyChain1} zeigt das Ergebnis f\"ur
%Gl. \refkl{supplyChainTwo} f\"ur den Fall, dass die Nachfrage
%pl\"otzlich von 1 auf 2 steigt. Hier konnte die Nachfrage gerade noch
%zu jeder Zeit befriedigt werden; bei einem h\"oheren als 100\%-tigen
%Anstieg g\"abe es Knappheit. Solche Anstiege k\"onnten vom Anbieter  dadurch
%ber\"ucksichtigt werden, dass er $\tau_2/\tau_1$ erh\"oht: $\tau_2$
%gibt die Zeit an, in der sich ein auf dem Normf\"ullstand befindliches
%Lager ohne Nachschub bei gleichbleibender Nachfrage
%vollst\"andig leeren w\"urde ($x_2=-1$) und $\tau_1$ die Zeit, die
%der Produzent im Mittel ben\"otigt, um seine Produktion $x_1$ 
%hoch oder herunterzufahren.

























